\documentclass{article}
\usepackage[english]{babel}
\usepackage{longtable}
\usepackage[top=1in, bottom=0.25in, left=1.25in, right=1.25in,includefoot,heightrounded]{geometry}
\usepackage{indentfirst}
\usepackage[utf8]{inputenc}
\usepackage{amsmath,amssymb}
\usepackage{graphicx,tikz}
\usepackage{hyperref}
\usepackage[colorinlistoftodos]{todonotes}
\usepackage[document]{ragged2e}
\usepackage{fancyhdr}
\usepackage{enumerate}
\usepackage{listings}
\usepackage{color}
\usepackage{flowchart}
\usepackage{hyperref}
\usepackage{graphicx}
\usetikzlibrary{arrows}

\usetikzlibrary{shapes.geometric, arrows}
\tikzstyle{startstop} = [rectangle, rounded corners, minimum width=3cm, minimum height=1cm,text centered, draw=black, fill=red!30]
\tikzstyle{decision} = [diamond, minimum width=4cm, minimum height=0.5cm, text centered, draw=black, fill=green!30]
\tikzstyle{process} = [rectangle, minimum width=3cm, minimum height=1cm, text centered, draw=black, fill=orange!30]
\tikzstyle{arrow} = [thick,->,>=stealth]
\tikzstyle{io} = [trapezium, trapezium left angle=70, trapezium right angle=110, minimum width=2cm, text width=4cm, minimum height=1cm, text centered, draw=black, fill=blue!30]

\pagestyle{fancy}
\fancyhf{}
\lhead{Myles Deslippe}
\rhead{Comp 3150 | Database Management Systems}
\cfoot{\thepage}

\definecolor{MyDarkGreen}{rgb}{0.0,0.4,0.0}
\lstset{inputencoding=ansinew}
\lstset{breaklines=true} 

\begin{document}

    \section*{\centering{Introduction}}

    \subsection*{Basic Definitions}
    \begin{itemize}
        \item The term \textbf{data} refers to an \textbf{individual unit of information}.
        \item The term \textbf{record} refers to a \textbf{grouping of data}.
        \item The term \textbf{database} refers to an \textbf{organized collection of records}.
        \item \textbf{Databases} are \textbf{managed} by \textbf{software systems} known as \textbf{Database Management Systems (DBMS)}.
        \begin{itemize}
            \item These systems allow \textbf{users} to \textbf{manage the database along with the data stored within the database}.
        \end{itemize}
    \end{itemize}

    \subsection*{Typical DBMS Functionality}
    \begin{itemize}
        \item \textbf{Database Management Systems} provide a \textbf{data model} that is used to \textbf{define the layout of databases}.
        \begin{itemize}
            \item \textbf{Data models} are composed of \textbf{data types, structures, and constraints}.
            \item There are a large amount of data models, some of the more popular ones include: the relational model, the document model, and the network model.
        \end{itemize}
        \item \textbf{Database Management Systems} provide a \textbf{Data Definition Language (DDL)}, which is a \textbf{domain-specific language} that allows \textbf{users} to \textbf{create, read, update, and delete} the \textbf{structure (schema)} of a \textbf{database}.
        \item \textbf{Database Management Systems} provide a \textbf{Data Manipulation Language (DML)}, which is a \textbf{domain-specific language} that allows \textbf{users} to \textbf{create, read, update, and delete} the \textbf{records} stored within the \textbf{databases}.
        \item The \textbf{DDL and DML} are sometimes \textbf{combined} into a \textbf{single-language}.
        \item \textbf{Database Management Systems} allow \textbf{concurrent processing} and \textbf{sharing of data} to \textbf{external users and external applications}.
        \begin{itemize}
            \item This is done with the use of \textbf{Integrity Constraints (IC)} and \textbf{transactions} to ensure the \textbf{data} is always \textbf{valid and consistent}.
        \end{itemize}
        \item \textbf{Databases Management Systems} provide \textbf{security tools} that \textbf{prevent unauthorized access} to the \textbf{data} stored within the \textbf{databases}.
        \item \textbf{Database Management Systems} provide \textbf{maintenance tools} that help administrators \textbf{maintain databases} over the \textbf{lifetime} of \textbf{applications}.
    \end{itemize}

    \subsection*{Characteristics of Databases}
    \begin{itemize}
        \item \textbf{Databases} are naturally \textbf{self-describing}. The \textbf{Database Management System} stores a \textbf{catalog of meta-data} that describes \textbf{databases} (types, constraints, etc).
        \item \textbf{Databases} provide a \textbf{layer of insulation} between \textbf{external applications} that \textbf{access the same datasets}.
        \begin{itemize}
            \item This is know as \textbf{program-data independence}; it is made possible by the \textbf{three-level architecture design} that databases use.
        \end{itemize}
        \item The \textbf{physical storage} of \textbf{data} is \textbf{abstracted} from \textbf{users and applications}; they only know about the \textbf{external view} of the \textbf{database}.
        \item \textbf{Users and programs} refer to the \textbf{data model constructs} rather than the \textbf{internal storage details}.
        \item The \textbf{Database Management System} supports \textbf{multiple-views} of \textbf{databases} that are based on a \textbf{per-user / per-application permission system}.
        \item When \textbf{users and applications} are interacting with the \textbf{Database Management System}, they do so without having to be \textbf{concerned with interfering with each other}, the \textbf{DBMS automatically handles this}.
    \end{itemize}
    
    \subsection*{Database Advantages Over the File System}
    \begin{itemize}
        \item \textbf{Databases store all data in a single place}, so there is \textbf{no need for redundant copies} of the \textbf{dataset}.
        \item \textbf{Databases} support \textbf{complex, fine-grained, access-controls}.
        \item \textbf{Databases} support \textbf{non-primitive storage types}.
        \item \textbf{Databases} automatically \textbf{optimize queries for you}.
        \item \textbf{Databases} support \textbf{Integrity Constraints}.
    \end{itemize}
    
    \subsection*{Types of Users}
    \begin{itemize}
        \item Database Administrators (DBA).
        \item Database Designers.
        \item End-Users.
        \item Programmers.
        \item Developers of the DBMS.
    \end{itemize}

\end{document}
