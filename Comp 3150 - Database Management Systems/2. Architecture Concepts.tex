\documentclass{article}
\usepackage[english]{babel}
\usepackage{longtable}
\usepackage[top=1in, bottom=0.25in, left=1.25in, right=1.25in,includefoot,heightrounded]{geometry}
\usepackage{indentfirst}
\usepackage[utf8]{inputenc}
\usepackage{amsmath,amssymb}
\usepackage{graphicx,tikz}
\usepackage{hyperref}
\usepackage[colorinlistoftodos]{todonotes}
\usepackage[document]{ragged2e}
\usepackage{fancyhdr}
\usepackage{enumerate}
\usepackage{listings}
\usepackage{color}
\usepackage{flowchart}
\usepackage{hyperref}
\usepackage{graphicx}
\usetikzlibrary{arrows}

\usetikzlibrary{shapes.geometric, arrows}
\tikzstyle{startstop} = [rectangle, rounded corners, minimum width=3cm, minimum height=1cm,text centered, draw=black, fill=red!30]
\tikzstyle{decision} = [diamond, minimum width=4cm, minimum height=0.5cm, text centered, draw=black, fill=green!30]
\tikzstyle{process} = [rectangle, minimum width=3cm, minimum height=1cm, text centered, draw=black, fill=orange!30]
\tikzstyle{arrow} = [thick,->,>=stealth]
\tikzstyle{io} = [trapezium, trapezium left angle=70, trapezium right angle=110, minimum width=2cm, text width=4cm, minimum height=1cm, text centered, draw=black, fill=blue!30]

\pagestyle{fancy}
\fancyhf{}
\lhead{Myles Deslippe}
\rhead{Comp 3150 | Database Management Systems}
\cfoot{\thepage}

\definecolor{MyDarkGreen}{rgb}{0.0,0.4,0.0}
\lstset{inputencoding=ansinew}
\lstset{breaklines=true} 

\begin{document}

    \section*{\centering{Architecture Concepts}}

    \subsection*{Data Models}
    \begin{itemize}
        \item \textbf{Data Models} are used to define the \textbf{structure, operations, and integrity constraints} that are \textbf{supported}.
        \begin{itemize}
            \item There are a large amount of data models, some of the more popular ones include: the relational model, the document model, and the network model.
        \end{itemize}
        \item The \textbf{data model} that should be used to \textbf{solve a problem} is determined by the \textbf{specific requirements of the problem}.
        \item There are \textbf{four categories} of \textbf{data models}:
        \begin{enumerate}
            \item \textbf{Conceptual, high-level, data models} | These \textbf{describe data} in concepts that are \textbf{close to the way} that \textbf{external users perceive the data}.
            \item \textbf{Physical, low-level, data models} | These \textbf{describe data} in concepts that \textbf{detail how the data is physically stored}.
            \item \textbf{Implementation, representation, data models} | These \textbf{describe data} in concepts that are a \textbf{hybrid of (1) and (2)}. There is some information regarding the way the data is stored, but overall it is still very high-level.
            \item \textbf{Self-Describing data models} | These \textbf{describe how data is being stored, near the actual data that is being stored}.
        \end{enumerate}
    \end{itemize}

    \subsection*{Schemas vs Instances}
    \begin{itemize}
        \item A \textbf{schema} is a \textbf{description of the data that is being stored} (typically referring to the entire database).
        \item A \textbf{schema construct} is a \textbf{sub-component} of the \textbf{schema} (for example a table).
        \item An \textbf{instance} is a \textbf{record that is actually being stored}.
        \item The \textbf{state of a database} refers to \textbf{all of the records stored} within the \textbf{database} at a \textbf{particular moment in time}.
        \item The act of \textbf{saving the database state} is called \textbf{creating a snapshot of the database}.
        \item The \textbf{database state} is said to be \textbf{valid} when all of the \textbf{data} that is \textbf{stored within the database} satisfies the \textbf{structure and constraints of the schema}.
        \item The \textbf{schema of a database} can also be referred to as \textbf{"intension"}.
        \item The \textbf{state of a database} can also be referred to as \textbf{"extension"}.
    \end{itemize}

    \subsection*{Three-Level Architecture}
    \begin{itemize}
        \item \textbf{Database Management Systems} use a \textbf{three-level architecture} to achieve: \textbf{data-program separation, data-independence, and multi-view support}.
        \item The three levels are defined as follows:
        \begin{enumerate}
            \item The \textbf{first schema} is an \textbf{external schema} that is used for \textbf{user views}.
            \item The \textbf{second schema} is a \textbf{conceptual schema} that models the \textbf{entire database} with a \textbf{data model}.
            \item The \textbf{third schema} is an \textbf{internal schema} that models the \textbf{physical data storage}.
        \end{enumerate}
    \end{itemize}

    \subsection*{Data Independence}
    \begin{itemize}
        \item The term \textbf{data independence} refers to the ability to change a \textbf{lower-level schema without having to modify any higher-level schemas}.
        \item There are \textbf{two types of data independence}:
        \begin{enumerate}
            \item \textbf{Logical Data Independence} is the ability to change the \textbf{conceptual schema} without having to change the \textbf{external schema}.
            \item \textbf{Physical Data Independence} is the ability to change the \textbf{internal schema} without having to change the \textbf{conceptual schema}.
        \end{enumerate}
    \end{itemize}

\end{document}
