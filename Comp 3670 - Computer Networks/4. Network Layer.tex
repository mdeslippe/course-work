\documentclass[16pt]{article}
\usepackage[english]{babel}
\usepackage{longtable}
\usepackage[top=1in, bottom=0.25in, left=1.25in, right=1.25in,includefoot,heightrounded]{geometry}
\usepackage{indentfirst}
\usepackage[utf8]{inputenc}
\usepackage{amsmath,amssymb}
\usepackage{graphicx,tikz}
\usepackage{hyperref}
\usepackage[colorinlistoftodos]{todonotes}
\usepackage[document]{ragged2e}
\usepackage{fancyhdr}
\usepackage{enumerate}
\usepackage{listings}
\usepackage{color}
\usepackage{flowchart}
\usepackage{hyperref}
\usepackage{graphicx}
\usetikzlibrary{arrows}

\usetikzlibrary{shapes.geometric, arrows}
\tikzstyle{startstop} = [rectangle, rounded corners, minimum width=3cm, minimum height=1cm,text centered, draw=black, fill=red!30]
\tikzstyle{decision} = [diamond, minimum width=4cm, minimum height=0.5cm, text centered, draw=black, fill=green!30]
\tikzstyle{process} = [rectangle, minimum width=3cm, minimum height=1cm, text centered, draw=black, fill=orange!30]
\tikzstyle{arrow} = [thick,->,>=stealth]
\tikzstyle{io} = [trapezium, trapezium left angle=70, trapezium right angle=110, minimum width=2cm, text width=4cm, minimum height=1cm, text centered, draw=black, fill=blue!30]

\pagestyle{fancy}
\fancyhf{}
\lhead{Myles Deslippe}
\rhead{Comp 3670 | Computer Networks}
\cfoot{\thepage}

\definecolor{MyDarkGreen}{rgb}{0.0,0.4,0.0}
\lstset{inputencoding=ansinew}
\lstset{breaklines=true} 

\begin{document}

\section*{\centering{The Network Layer}}

    \subsection*{Services and Protocols}
    \begin{itemize}
        \item To transport \textbf{segments} from the \textbf{sending host}, to the \textbf{receiving host} the following happens:
        \begin{enumerate}
            \item The \textbf{sender} \textbf{encapsulates segments} into \textbf{datagrams} and passes them to the \textbf{link layer}.
            \item The \textbf{receiver delivers segments} to the \textbf{transport layer protocol}. 
            \item A \textbf{router} is a piece of \textbf{network hardware} than manages \textbf{traffic between networks}.
            \begin{itemize}
                \item Routers work by examining the headers in \textbf{IP datagrams (Packets)}, and move the datagrams from \textbf{input ports} to \textbf{output ports}; with the goal of transfering datagrams along the end-end path.
                \item Routers work a the \textbf{Network Layer (Layer 3)}, and also use layers 1 and 2 to facilitate the data transfer.
                \item Routers us \textbf{Internet Protocol Addresses (IP Address)} to identify networks / hosts.
            \end{itemize}
            \item  
        \end{enumerate}
    \end{itemize}

    \subsection*{Key Network-Layer Functions}
    \begin{itemize}
        \item One key network-layer function is \textbf{forwarding}, \textbf{forwarding} involves \textbf{moving packets} from a \textbf{rounter's input link} to the appropriate \textbf{output link}.
        \item Another key network-layer function is \textbf{routing}, \textbf{routing} involves \textbf{determining the route taken by packets} from the \textbf{source} to the \textbf{destination}.
        \begin{itemize}
            \item There are many routing algorithms that can be used the achieve this.
        \end{itemize}
        \item 
    \end{itemize}

    \subsection*{The Data Plane vs The Control Plane}
    \begin{itemize}
        \item The \textbf{data plane} is a \textbf{local, per-router function} that \textbf{determines how packets} arriving on a router's input port \textbf{is forwarded to router's output port}.
        \item The \textbf{control plane} is a \textbf{network-wide} function, that \textbf{determines how packets are routed amongst routers} along end-end paths from \textbf{source host} to \textbf{destination host}.
        \begin{itemize}
            \item There are two control-plane approaches:
            \begin{enumerate}
                \item \textbf{Traditional routing algorithms} that are implemented in routers.
                \item \textbf{Software-defined networking (SDN)} that is implemented in remote servers.
            \end{enumerate}
        \end{itemize}
    \end{itemize}

    \subsection*{Per-Router Control Plane}
    \begin{itemize}
        \item One way to implemented a \textbf{control-plane routing algorithm} is in \textbf{every router}. Each router determines the \textbf{next, best, destination}. 
    \end{itemize}

\end{document}