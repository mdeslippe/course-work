\documentclass{article}
\usepackage[english]{babel}
\usepackage{longtable}
\usepackage[top=1in, bottom=0.25in, left=1.25in, right=1.25in,includefoot,heightrounded]{geometry}
\usepackage{indentfirst}
\usepackage[utf8]{inputenc}
\usepackage{amsmath,amssymb}
\usepackage{graphicx,tikz}
\usepackage{hyperref}
\usepackage[colorinlistoftodos]{todonotes}
\usepackage[document]{ragged2e}
\usepackage{fancyhdr}
\usepackage{enumerate}
\usepackage{listings}
\usepackage{color}
\usepackage{flowchart}
\usepackage{hyperref}
\usepackage{graphicx}
\usetikzlibrary{arrows}

\usetikzlibrary{shapes.geometric, arrows}
\tikzstyle{startstop} = [rectangle, rounded corners, minimum width=3cm, minimum height=1cm,text centered, draw=black, fill=red!30]
\tikzstyle{decision} = [diamond, minimum width=4cm, minimum height=0.5cm, text centered, draw=black, fill=green!30]
\tikzstyle{process} = [rectangle, minimum width=3cm, minimum height=1cm, text centered, draw=black, fill=orange!30]
\tikzstyle{arrow} = [thick,->,>=stealth]
\tikzstyle{io} = [trapezium, trapezium left angle=70, trapezium right angle=110, minimum width=2cm, text width=4cm, minimum height=1cm, text centered, draw=black, fill=blue!30]

\pagestyle{fancy}
\fancyhf{}
\lhead{Myles Deslippe}
\rhead{Comp 3670 | Computer Networks}
\cfoot{\thepage}

\definecolor{MyDarkGreen}{rgb}{0.0,0.4,0.0}
\lstset{inputencoding=ansinew}
\lstset{breaklines=true} 

\begin{document}

    \section*{\centering{Network Security}}

    \textbf{Network Security Introduction}
    \begin{itemize}
        \item \textbf{Network security} consists of the \textbf{policies, processes, and practices adopted} to \textbf{prevent, detect, and monitor unathorized access, misuse, modification, or denial of service}.
        \item \textbf{Confidentiality} refers to the \textbf{state of keeping secret or private}. Only the \textbf{sender} and \textbf{intended reciever} of a \textbf{message} should be able to \textbf{understand the message}.
        \begin{itemize}
            \item This is done with \textbf{encryption}. The sender encrypts the message, and the inteneded receiver decrypts the message.
        \end{itemize}
        \item \textbf{Authentication} is the \textbf{process or action} of \textbf{verifying the identity} of a \textbf{user or process}.
        \begin{itemize}
            \item In \textbf{network security}, we want to \textbf{confirm the identity} of the \textbf{sender and receiver of messags}.
        \end{itemize}
        \item \textbf{Message integrity} means that a \textbf{message has not been tampered with or altered without detection}.
        \item \textbf{Access and availability of services} means tat services must be accessable and available to users.
    \end{itemize}

    \subsection*{Types of Actions a malicious Actors can Perform}
    \begin{itemize}
        \item \textbf{Malicious actors} can do the following (and other stuff that is not included):
        \begin{enumerate}
            \item \textbf{Eavesdrop / intercept messages}.
            \item \textbf{Actively insert messages} into a \textbf{connection}.
            \item \textbf{Impersonation via spoofing}.
            \item \textbf{Hijacking ongoing connections}.
            \item \textbf{Denial of service attacks}. 
        \end{enumerate}
    \end{itemize}

    \subsection*{Cryptography}
    \begin{itemize}
        \item To \textbf{encrypt a message}, the sender uses an \textbf{encryption key} to create a \textbf{ciphertext}, that can only be \textbf{decrypted by the intedned receiver's decryption key}.
        \item This way, \textbf{malicious actors} can view the \textbf{ciphertext}, but have to way to \textbf{know what it means}.
        \item There are \textbf{type classes of cryptographic algorithms}:
        \begin{enumerate}
            \item \textbf{Symmetric algorithms} use the \textbf{same key to encrypt and decrypt the message}.
            \item \textbf{Asymmetric algorithms} use \textbf{two different keys, one to encrypt and the other to decrypt the message}.
        \end{enumerate}
        \item To attempt to \textbf{decrypt ciphertext's}, \textbf{malicious actors} can use a \textbf{brute force search through all of the keys, or a statistical analysis}.
        \item If a \textbf{malicious actor} has \textbf{the plain text, and the ciphertext}, they can \textbf{determine the key}.
    \end{itemize}

    \subsection*{Cryptographic Algorithms}
    \begin{itemize}
        \item The \textbf{Data Encryption Standard (DES)} is an older encryption standard that uses a \textbf{56-bit symmetric key, with 64-bit input}.
        \begin{itemize}
            \item DES is not a good encryption algorithms as it can be brute forced in a day.
            \item One solution to improve the security, is to encrypt the input three times with three different keys (know as 3DES).
        \end{itemize}
        \item The \textbf{Advanced Encryption Standard (AES)} is a \textbf{symmetric encryption algorithm} created to \textbf{replace DES}. Keys can be \textbf{128-bit, 192-bit, or 256-bit}. The \textbf{encryped data} is \textbf{processes in 128-bit blocks}.
        \begin{itemize}
            \item Brute force decryption takes 149 trillion years for AES (whereas it takes 1 second with DES).
        \end{itemize}
        \item The \textbf{Rivest, Shamir, Adelson (RSA) algorithm} is an \textbf{asymmetric, public-private key encryption algorithm}.
    \end{itemize}

    \subsection*{Authentication}
    \begin{itemize}
        \item 
    \end{itemize}

\end{document}