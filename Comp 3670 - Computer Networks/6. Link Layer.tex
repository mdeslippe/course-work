\documentclass{article}
\usepackage[english]{babel}
\usepackage{longtable}
\usepackage[top=1in, bottom=0.25in, left=1.25in, right=1.25in,includefoot,heightrounded]{geometry}
\usepackage{indentfirst}
\usepackage[utf8]{inputenc}
\usepackage{amsmath,amssymb}
\usepackage{graphicx,tikz}
\usepackage{hyperref}
\usepackage[colorinlistoftodos]{todonotes}
\usepackage[document]{ragged2e}
\usepackage{fancyhdr}
\usepackage{enumerate}
\usepackage{listings}
\usepackage{color}
\usepackage{flowchart}
\usepackage{hyperref}
\usepackage{graphicx}
\usetikzlibrary{arrows}

\usetikzlibrary{shapes.geometric, arrows}
\tikzstyle{startstop} = [rectangle, rounded corners, minimum width=3cm, minimum height=1cm,text centered, draw=black, fill=red!30]
\tikzstyle{decision} = [diamond, minimum width=4cm, minimum height=0.5cm, text centered, draw=black, fill=green!30]
\tikzstyle{process} = [rectangle, minimum width=3cm, minimum height=1cm, text centered, draw=black, fill=orange!30]
\tikzstyle{arrow} = [thick,->,>=stealth]
\tikzstyle{io} = [trapezium, trapezium left angle=70, trapezium right angle=110, minimum width=2cm, text width=4cm, minimum height=1cm, text centered, draw=black, fill=blue!30]

\pagestyle{fancy}
\fancyhf{}
\lhead{Myles Deslippe}
\rhead{Comp 3670 | Computer Networks}
\cfoot{\thepage}

\definecolor{MyDarkGreen}{rgb}{0.0,0.4,0.0}
\lstset{inputencoding=ansinew}
\lstset{breaklines=true} 

\begin{document}

    \section*{\centering{The Link Layer}}

    \subsection*{Introduction to the Link Layer}
    \begin{itemize}
        \item The \textbf{link layer} is responsible for \textbf{transferring datagrams} from one \textbf{node} to another \textbf{physically adjacent} node over a \textbf{link}.
        \item When working at the \textbf{link layer}, we refer to \textbf{hosts and routers} as \textbf{nodes}.
        \item When working at the \textbf{link layer}, we refer to \textbf{communication channels} that \textbf{connect adjacent nodes} as \textbf{links}.
        \begin{itemize}
            \item These links can be wired and wireless.
        \end{itemize}
        \item When working at the \textbf{link layer}, \textbf{frames} are the \textbf{unit of data} we are interested in \textbf{transferring}; they \textbf{encapsulate packets}.
        \item Different \textbf{types of links} have \textbf{different transfer protocols}. Different protocols provice different services (eg one may be reliable, the other may not be reliable).
    \end{itemize}

    \subsection*{Link Layer Services}
    \begin{itemize}
        \item The \textbf{framing service} is responsible for \textbf{encapsulating datagrams} into frames (adding headers, etc). If there is a shared medium, the channel access needs to be specified. MAC addresses are used in frame headers to identify devices.
        \item The \textbf{reliable deliver service} is responsible for ensuring \textbf{every frame is correctly delivered}.
        \item The \textbf{flow control service} is responsible for \textbf{pacing sending and receiving rates}.
        \item The \textbf{error detection service} is responsible for \textbf{detecting errors} caused by frame drops, noise, signal attenuation, etc.
        \item The \textbf{error correction service} is responsible for \textbf{identifying incorrect bits, and correcting them} without the need for retransmission.
        \item With a \textbf{half-duplex service} both nodes can transmit and receive, but not both operations at the same time.
        \item With a \textbf{full-duplex service} both nodes can transmit and receive at the same time.
    \end{itemize}

    \subsection*{Link Layer Implementation}
    \begin{itemize}
        \item The \textbf{link layer} is \textbf{implemented} in \textbf{each and every node}.
        \item \textbf{Implementations} are located in the node's \textbf{Network Interface Card (NIC)}. These cards implement the link and physical layer.
        \item The \textbf{implementations} are \textbf{connected} to the \textbf{system's bus} which allow for data transfer. And a combination of \textbf{hardware, software, and firmware control the NIC}.
    \end{itemize}

    \subsection*{Error Detection}
    \begin{itemize}
        \item An \textbf{Error Correction and Detection (EDC) bit} is used.
        \item If an \textbf{error is detected}, the device will either \textbf{correct the error}, or request a \textbf{retransmission}.
        \item \textbf{Error detection} is not 100\% reliable, but it is still useful. Larger EDC fields result in better detection and correction.
        \item \textbf{Parity checking} is also used.
        \item \textbf{Cyclic Redundancy Check (CRC)} is a good way of \textbf{detecting errors}.
    \end{itemize}

    \subsection*{Multiple Access Protocols}
    \begin{itemize}
        \item There are two types of links:
        \begin{enumerate}
            \item \textbf{Point-To-Point} | Two devices connected directly by a link (for example ethernet between two devices).
            \item \textbf{Broadcast} | Several devices connected by a shared medium that can communicate (for example a shared wire, a shared radio, a switch).
            \item When \textbf{two or more} nodes \textbf{simultaneously transmit}, \textbf{interference can occur}.
            \item \textbf{Collision occurs} if a \textbf{single node} receives \textbf{two or more signals at the same time}.
            \item There are \textbf{protocols} to \textbf{avoid interference and collision}; they involve a \textbf{distributed algorithm} that determins \textbf{how nodes share a channel}.
            \item \textbf{Communication} about \textbf{how to use a channel} must be \textbf{on the channel}, out-of-band coordination is not allowed in these protocols.
        \end{enumerate}
    \end{itemize}

    \subsection*{The Media Access Control (MAC) Protocol}
    \begin{itemize}
        \item In the \textbf{MAC protocol}, there are \textbf{three classifications of channel access control}:
        \begin{enumerate}
            \item \textbf{Channel Partitioning} | Channels are divided into smaller pieces (time slots, requency, code, etc), and pieces of the channel are allocated to nodes for exclusive use.
            \item \textbf{Random Access} | Channels are not divided, and allow collision, and provides a way to recover from collisions.
            \item \textbf{Taking Turns} | Nodes take turns using the channel, but nodes with more to send can take longer turns.
        \end{enumerate}
    \end{itemize}

    \subsection*{MAC Channel Partitioning Protocols}
    \begin{itemize}
        \item \textbf{Time Division Multiple Access (TDMA)} gives nodes access to the channel in \textbf{rounds}, each node gets a \textbf{fixed length slot} (length = packet transmission time) in each round.
        \begin{itemize}
            \item Unused slots go idle.
        \end{itemize}
        \item \textbf{Frequency Division Multiple Access (FDMA)} divides the channel into \textbf{frequency bands}, and each nodes gets a \textbf{fixed frequency band}.
        \begin{itemize}
            \item Unused frequency bands go idle.
        \end{itemize}
    \end{itemize}

    \subsection*{MAC Channel Random Access Protocols}
    \begin{itemize}
        \item When a \textbf{node has a packet to send}, it transmits to the \textbf{full channel} at a data rate $R$ \textbf{without any prior coordination}.
        \item If \textbf{two nodes transmit} at the \textbf{same time}, \textbf{collision occurs}.
        \item The \textbf{Random Access protocol} specifies how to \textbf{detect and recover} from \textbf{collisions}.
        \item Exampels of \textbf{Random Access protocols}: 
        \begin{enumerate}
            \item \textbf{ALOHA}:
            \begin{itemize}
                \item Assumptions: 
                \begin{enumerate}
                    \item All frames are the same size.
                    \item Time is divided into equal size transmission slots.
                    \item Nodes can only start transmitting at a slot beginning.
                    \item The nodes are all time-synchronized.
                    \item If two or more nodes transmit in a slot, all nodes detect the collision.
                \end{enumerate}
                \item When a node obtains a fresh frame, it will transmit it in the next slot, if collision occurs, retransmit, otherwise send the next frame.
                \item Suppose N nodes with many frames to send, each transmit in a slot with probability $P$, each time a node attempts to transmit, 37\% ($\frac{100}{e}$\%) of the time, nodes will be able to transmit without collision.
                \item Pros:
                \begin{enumerate}
                    \item Single active node can continuously transmit at full rate.
                    \item Highly decentralized.
                    \item Simple.
                \end{enumerate}
                \item Cons:
                \begin{enumerate}
                    \item Collision occurs, wasting time slots.
                    \item Some slots are idle.
                    \item Nodes may be able to detect collision in lass than time to transmit a packet.
                    \item Clock synchronization is difficult.
                \end{enumerate}
            \end{itemize}
            \item \textbf{Pure ALOHA}:
            \begin{itemize}
                \item \textbf{Pure ALOHA} is \textbf{ALOHA without the timeslots}. When a frame first arrives, nodes can attempt to transmit it immediately.
                \item The probability for collision increases with no synchronization.
                \item Suppose N nodes with many frames to send, each transmit with probability $P$, each time a node attempts to transmit, 18\% ($\frac{100}{2e}$\%) of the time, nodes will be able to transmit without collision.
            \end{itemize}
            \item \textbf{Simple Carrier Sense Multiple Access (Simple CSMA)}:
            \begin{itemize}
                \item CSMA requires nodes to listen before they transmit. If the channel is idle, they can transmit the entire frame. If the channel is busy, they can defer the transmission.
            \end{itemize}
            \item \textbf{CSMA/CD}:
            \begin{itemize}
                \item \textbf{CSMA/CD} is \textbf{Simple CSMA} with \textbf{Collision Detection}.
                \item Collision can be detected within a short period of time.
                \item Colliding transmissions are aborted, and rescheduled, reducing channel waste.
                \item Collision detection is easy in wired links, but difficult in wireless links.
                \item \textbf{CSMA/CD} is more efficient than \textbf{ALOHA}.
            \end{itemize}
        \end{enumerate}
    \end{itemize}

    \subsection*{MAC Taking Turns Protocols}
    \begin{itemize}
        \item \textbf{Taking turns} uses \textbf{polling}, a \textbf{master} invites \textbf{other nodes to transmit in turn}.
        \item Concerns with this type of protocol are: polling overhead, latency, and a single point of failiure (master).
        \item Another way to implement taking turns is with \textbf{token passing}. A \textbf{control token} is \textbf{passed sequentially} from \textbf{one node to the next}.
        \item Concerns with token passing are: token overhead, latency, single point of failiure (token).
    \end{itemize}

    \section*{\centering{Local Area Networks (LANs)}}

    \subsection*{MAC Addresses}
    \begin{itemize}
        \item 
    \end{itemize}

\end{document}