\documentclass{article}
\usepackage[english]{babel}
\usepackage{longtable}
\usepackage[top=1in, bottom=0.25in, left=1.25in, right=1.25in,includefoot,heightrounded]{geometry}
\usepackage{indentfirst}
\usepackage[utf8x]{inputenc}
\usepackage{amsmath,amssymb}
\usepackage{graphicx,tikz}
\usepackage{hyperref}
\usepackage[colorinlistoftodos]{todonotes}
\usepackage[document]{ragged2e}
\usepackage{fancyhdr}
\usepackage{enumerate}
\usepackage{listings}
\usepackage{color}
\usepackage{flowchart}
\usepackage{hyperref}
\usepackage{graphicx}
\usetikzlibrary{arrows}

\usetikzlibrary{shapes.geometric, arrows}
\tikzstyle{startstop} = [rectangle, rounded corners, minimum width=3cm, minimum height=1cm,text centered, draw=black, fill=red!30]
\tikzstyle{decision} = [diamond, minimum width=4cm, minimum height=0.5cm, text centered, draw=black, fill=green!30]
\tikzstyle{process} = [rectangle, minimum width=3cm, minimum height=1cm, text centered, draw=black, fill=orange!30]
\tikzstyle{arrow} = [thick,->,>=stealth]
\tikzstyle{io} = [trapezium, trapezium left angle=70, trapezium right angle=110, minimum width=2cm, text width=4cm, minimum height=1cm, text centered, draw=black, fill=blue!30]

\pagestyle{fancy}
\fancyhf{}
\lhead{Myles Deslippe}
\rhead{Comp 3670 | Computer Networks}
\cfoot{\thepage}

\definecolor{MyDarkGreen}{rgb}{0.0,0.4,0.0}
\lstset{inputencoding=ansinew}
\lstset{breaklines=true} 

\begin{document}

    \section*{The Control Plane}

    \subsection*{Network-Layer Functions}
    \begin{itemize}
        \item The \textbf{data plane} is reponsible for \textbf{forwarding packets} (moving them from the router's input, to the router's output).
        \item The \textbf{control plane} is responsible for \textbf{determining the route} taken by \textbf{packets} from \textbf{source to destination}. 
    \end{itemize}

    \subsection*{Structuring the Control Plane}
    \begin{itemize}
        \item There are two ways to \textbf{structure the network control plane}:
        \begin{enumerate}
            \item \textbf{Per-router control (traditional)} | Each \textbf{router} has a \textbf{routing algorithm} that is used to determine \textbf{where to route the packet}.
            \item \textbf{Logically centralized control (software defined networking)} | Remote controller computes, and \textbf{installs a fordwarding table} in the \textbf{routers}.
        \end{enumerate}
    \end{itemize}

    \section*{\centering{Routing Protocols}}
    
    \subsection*{Routing Protocols}
    \begin{itemize}
        \item The \textbf{goal} of a \textbf{routing protocol} is to \textbf{determine "good" routes} from the \textbf{sending host} to the \textbf{receiving host} through a network of routers.
        \item In order to acheive that goal, \textbf{each router} needs to know \textbf{what it is directly conntected to}, and what \textbf{those routers are connected to}.
        \item A \textbf{path} is a sequence of \textbf{routers} that packets must \textbf{traverse} from the \textbf{inital sending host} to the \textbf{final destination host}.
        \item A \textbf{"good" route} is a \textbf{route} that is the \textbf{fastest, least congested, and of least "cost"}.
    \end{itemize}

    \subsection*{Routing Graphs}
    \begin{itemize}
        \item A \textbf{routing graph} is a tuple $G=(N, E)$ where $N$ is a set of routers $\{n_1, n_2, \cdots, n_j\}$ and $E$ is a set of links $\{e_1, e_2, \cdots, e_k\}$.
        \item The \textbf{cost} of a \textbf{link} $l\in{E}$ is defined as a function $C: E\rightarrow\mathbb{R}\cup\{\infty\}$, denoted by $C_{a,b}$ where $a,b\in{N}$ are the routers that the link $l$ is contected to.
    \end{itemize}

    \subsection*{Routing Algorithms}
    \begin{itemize}
        \item A \textbf{routing algorithm} is an \textbf{algorithm} that is used to determine the \textbf{a "good" path} that a \textbf{packet} should take to get from \textbf{a sending host} to a \textbf{receiving host}.
        \item \textbf{Route classificatons}:
        \begin{enumerate}
            \item \textbf{Static Routes} | Static routes are routes that do \textbf{do not change}, or that \textbf{change very slowly over time}.
            \item \textbf{Dynamic Routes} | Dynamic routers are routes that \textbf{change quickly over time}, or have a \textbf{quickly chaning cost}.
        \end{enumerate}
        \item \textbf{Routing algorithm classifications}:
        \begin{enumerate}
            \item \textbf{Link State Algorithms (Global)} | Link state algorithms are used when \textbf{all routers} have a \textbf{complete topology of the network, and know the cost of each route}.
            \begin{itemize}
                \item An example of link state algorithms is Dijkstra's link-state routing algorithm.
            \end{itemize}
            \item \textbf{Distance Vector Algorithms (Decentralized)} | Distance vector algorithms are used routers initially \textbf{only know the link cost to attached neighbors}. This algorithm is \textbf{iterative}, and information needs to be \textbf{exchanged with neighboring routers}.
        \end{enumerate}
    \end{itemize}

    \subsection*{Dijkstra's Link-State Routing Algorithm}
    \begin{itemize}
        \item Notations:
        \begin{enumerate}
            \item $C_{x,y}$ | The direct link cost from node $x$ to node $y$. If $x$ and $y$ are not directly connected, $C_{x,y}=\infty$.
            \item $D(v)$ | The current least-cost-path cost estimate from source to destination $v$.
            \item $p(v)$ | Predecessor node along path from source to $v$.
            \item $N'$ | The set of nodes whose least-cost-path is definitively known.
        \end{enumerate}
        \item The algorithm:
        \begin{verbatim}
    begin
        N' = {u}
        for all nodes v
            if v is adjacent to u
                D(v) = Cu,v
            else
                D(v) = ∞
        
        Loop
            find w net in N' such that D(w) is minimum
            add w to N'
            update D(v) for all v adjacent to w and not in N'
            D(v) = min( D(v), D(w) + Cw,v )
        Until all nodes are in N'
    end
        \end{verbatim}
        \item The complexity of this algorithm is $\text{O}(n^2)$.
        \item There are more efficient implementations that are $\text{O}(n\log{n})$.
    \end{itemize}

\end{document}