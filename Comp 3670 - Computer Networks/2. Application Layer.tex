\documentclass[16pt]{article}
\usepackage[english]{babel}
\usepackage{longtable}
\usepackage[top=1in, bottom=1.25in, left=1.25in, right=1.25in,includefoot,heightrounded]{geometry}
\usepackage{indentfirst}
\usepackage[utf8]{inputenc}
\usepackage{amsmath,amssymb}
\usepackage{graphicx,tikz}
\usepackage{hyperref}
\usepackage[colorinlistoftodos]{todonotes}
\usepackage[document]{ragged2e}
\usepackage{fancyhdr}
\usepackage{enumerate}
\usepackage{listings}
\usepackage{color}
\usepackage{flowchart}
\usepackage{hyperref}
\usepackage{graphicx}
\usetikzlibrary{arrows}


\usetikzlibrary{shapes.geometric, arrows}
\tikzstyle{startstop} = [rectangle, rounded corners, minimum width=3cm, minimum height=1cm,text centered, draw=black, fill=red!30]
\tikzstyle{decision} = [diamond, minimum width=4cm, minimum height=0.5cm, text centered, draw=black, fill=green!30]
\tikzstyle{process} = [rectangle, minimum width=3cm, minimum height=1cm, text centered, draw=black, fill=orange!30]
\tikzstyle{arrow} = [thick,->,>=stealth]
\tikzstyle{io} = [trapezium, trapezium left angle=70, trapezium right angle=110, minimum width=2cm, text width=4cm, minimum height=1cm, text centered, draw=black, fill=blue!30]

\pagestyle{fancy}
\fancyhf{}
\lhead{Myles Deslippe}
\rhead{Comp 3670 | Computer Networks}
\cfoot{\thepage}

\renewcommand{\headrulewidth}{0.4pt}
\renewcommand{\baselinestretch}{1.2}
\renewcommand{\labelitemi}{$\circ$}

\definecolor{MyDarkGreen}{rgb}{0.0,0.4,0.0}
\lstset{inputencoding=ansinew}
\lstset{breaklines=true} 

\begin{document}

\section*{\centering{Network Applications}}

    \subsection*{Principals of Network Applications}
    \begin{itemize}
        \item To create a \textbf{network application}, we need to write a program that runs on \textbf{different end systems}, and \textbf{communicates over a network}.
        \item There is a \textbf{layer of abstraction} between \textbf{network applications} and \textbf{the network}; allowing for \textbf{rapid network application development}.
        \item There are different \textbf{application architectures} we can use to developer \textbf{network applications}.
    \end{itemize}

    \subsection*{Client-Server Architecture}
    \begin{itemize}
        \item The \textbf{client-server architecture} consists of two entities, the \textbf{client} and the \textbf{server}.
        \item The \textbf{sever} is a \textbf{network host that is always on}, and has a \textbf{permanent IP address}.
        \item The \textbf{client} communicate with the \textbf{server} over a \textbf{network}, and can have a \textbf{dynamic IP address}.
        \item The clients do not \textbf{directly communicate}, they use the \textbf{server to communicate}.
    \end{itemize}

    \subsection*{Peer to Peer Architecture}
    \begin{itemize}
        \item The \textbf{peer to peer architecture} has \textbf{no always-on server}.
        \item The \textbf{clients communicate directly}.
    \end{itemize}

\end{document}