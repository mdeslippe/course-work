\documentclass[16pt]{article}
\usepackage[english]{babel}
\usepackage{longtable}
\usepackage[top=1in, bottom=0.25in, left=1.25in, right=1.25in,includefoot,heightrounded]{geometry}
\usepackage{indentfirst}
\usepackage[utf8]{inputenc}
\usepackage{amsmath,amssymb}
\usepackage{graphicx,tikz}
\usepackage{hyperref}
\usepackage[colorinlistoftodos]{todonotes}
\usepackage[document]{ragged2e}
\usepackage{fancyhdr}
\usepackage{enumerate}
\usepackage{listings}
\usepackage{color}
\usepackage{flowchart}
\usepackage{hyperref}
\usepackage{graphicx}
\usetikzlibrary{arrows}

\usetikzlibrary{shapes.geometric, arrows}
\tikzstyle{startstop} = [rectangle, rounded corners, minimum width=3cm, minimum height=1cm,text centered, draw=black, fill=red!30]
\tikzstyle{decision} = [diamond, minimum width=4cm, minimum height=0.5cm, text centered, draw=black, fill=green!30]
\tikzstyle{process} = [rectangle, minimum width=3cm, minimum height=1cm, text centered, draw=black, fill=orange!30]
\tikzstyle{arrow} = [thick,->,>=stealth]
\tikzstyle{io} = [trapezium, trapezium left angle=70, trapezium right angle=110, minimum width=2cm, text width=4cm, minimum height=1cm, text centered, draw=black, fill=blue!30]

\pagestyle{fancy}
\fancyhf{}
\lhead{Myles Deslippe}
\rhead{Phil 2210 | Introduction to Ethics}
\cfoot{\thepage}

\definecolor{MyDarkGreen}{rgb}{0.0,0.4,0.0}
\lstset{inputencoding=ansinew}
\lstset{breaklines=true} 

\begin{document}

    \section*{\centering{The Nature of Moral Value}}

    \subsection*{The Nature of Moral Value}
    \begin{itemize}
        \item \textbf{J. L. Mackie} (a moral anti-realist) believes that there are \textbf{no universal more values} based on \textbf{relativistic arguments}.
        \item \textbf{Mackie} argues for what he calls \textbf{moral skepticism}, which is the view that \textbf{there are no objective values}.
        \begin{itemize}
            \item This is distinguished from \textbf{moral objectivism} which involves  the view that \textbf{there are objective values}.
        \end{itemize}
        \item Before his arguments, he makes a couple of distinctions first:
        \begin{enumerate}
            \item He distinguishes moral skepticism from the following views:
            \begin{enumerate}
                \item Our current system of values needs to be replaced with a better system of values.
                \item No existing system of moral values is adequate.
            \end{enumerate}
            \item He distinguishes second-order mmoral questions from first-order moral question.
            \begin{enumerate}
                \item First-order moral questions are concerned with moral values.
                \item Second-order moral questions are concerned with the possible existence of objective moral values, independently of any system of values.
            \end{enumerate}
            \item He distinguishes moral skepticism from simple subjectivism.
            \item He distinguishes the thesis that \textbf{there are objective moral values} fomr the thesis that \textbf{there are inter-subjective values}.
            \begin{itemize}
                \item One could be a moral skeptic while still believing that there are inner-subjective values shared across both individuals and cultures.
                \item These inter subjective values may still not be objective in the sense that they are true for all time.
            \end{itemize}
            \item 
        \end{enumerate}
        \item 
    \end{itemize}

    \subsection*{Thomas Nagel}
    \begin{itemize}
        \item \textbf{Thomas Nagel} (a moral realist) admits that there is \textbf{atleast one very general value} that is \textbf{objective}.
        \item 
    \end{itemize}

\end{document}