\documentclass{article}
\usepackage[english]{babel}
\usepackage{longtable}
\usepackage[top=1in, bottom=0.25in, left=1.25in, right=1.25in,includefoot,heightrounded]{geometry}
\usepackage{indentfirst}
\usepackage[utf8]{inputenc}
\usepackage{amsmath,amssymb}
\usepackage{graphicx,tikz}
\usepackage{hyperref}
\usepackage[colorinlistoftodos]{todonotes}
\usepackage[document]{ragged2e}
\usepackage{fancyhdr}
\usepackage{enumerate}
\usepackage{listings}
\usepackage{color}
\usepackage{flowchart}
\usepackage{hyperref}
\usepackage{graphicx}
\usetikzlibrary{arrows}

\usetikzlibrary{shapes.geometric, arrows}
\tikzstyle{startstop} = [rectangle, rounded corners, minimum width=3cm, minimum height=1cm,text centered, draw=black, fill=red!30]
\tikzstyle{decision} = [diamond, minimum width=4cm, minimum height=0.5cm, text centered, draw=black, fill=green!30]
\tikzstyle{process} = [rectangle, minimum width=3cm, minimum height=1cm, text centered, draw=black, fill=orange!30]
\tikzstyle{arrow} = [thick,->,>=stealth]
\tikzstyle{io} = [trapezium, trapezium left angle=70, trapezium right angle=110, minimum width=2cm, text width=4cm, minimum height=1cm, text centered, draw=black, fill=blue!30]

\pagestyle{fancy}
\fancyhf{}
\lhead{Myles Deslippe}
\rhead{Phil 2210 | Introduction to Ethics}
\cfoot{\thepage}

\definecolor{MyDarkGreen}{rgb}{0.0,0.4,0.0}
\lstset{inputencoding=ansinew}
\lstset{breaklines=true} 

\begin{document}

    \section*{\centering{The Nature of Moral Value}}

    \subsection*{The Nature of Moral Value}
    \begin{itemize}
        \item \textbf{J. L. Mackie} (a moral anti-realist) believes that there are \textbf{no universal more values} based on \textbf{relativistic arguments}.
        \item \textbf{Mackie} argues for what he calls \textbf{moral skepticism}, which is the view that \textbf{there are no objective values}.
        \begin{itemize}
            \item This is distinguished from \textbf{moral objectivism} which involves  the view that \textbf{there are objective values}.
        \end{itemize}
        \item Before his arguments, he makes a couple of distinctions first:
        \begin{enumerate}
            \item He distinguishes moral skepticism from the following views:
            \begin{enumerate}
                \item Our current system of values needs to be replaced with a better system of values.
                \item No existing system of moral values is adequate.
            \end{enumerate}
            \item He distinguishes second-order mmoral questions from first-order moral question.
            \begin{enumerate}
                \item First-order moral questions are concerned with moral values.
                \item Second-order moral questions are concerned with the possible existence of objective moral values, independently of any system of values.
            \end{enumerate}
            \item He distinguishes moral skepticism from simple subjectivism.
            \item He distinguishes the thesis that \textbf{there are objective moral values} fomr the thesis that \textbf{there are inter-subjective values}.
            \begin{itemize}
                \item One could be a moral skeptic while still believing that there are inner-subjective values shared across both individuals and cultures.
                \item These inter subjective values may still not be objective in the sense that they are true for all time.
            \end{itemize}
        \end{enumerate}
        \item Mackie proposed the following two arguments for his \textbf{moral skepticism position}:
        \begin{enumerate}
            \item The argument from the ontological (what exists) and epistemological (what we know) \textbf{weirdness of objective moral values}.
            \begin{itemize}
                \item Ontological weirdness: If there are objective moral values, they are not visible nor are they tangible. What kind of things are they? How do we know they exist?
                \item Epistemological weirdness: Even if there were a plausible ontological account of objective moral values, how could we come to know them?
            \end{itemize}
            \item The argument from cultural relativism: \newline Premise | Societies differ markedly in their moral values. \newline Conclusion | There are no objective moral values.
        \end{enumerate}
    \end{itemize}

    \subsection*{Thomas Nagel}
    \begin{itemize}
        \item \textbf{Thomas Nagel} (a moral realist) admits that there is \textbf{atleast one very general value} that is \textbf{objective}.
        \item \textbf{Thomas Nagel}'s goal is to find \textbf{at least one objective moral value} thereby contradicting \textbf{Mackie's moral skepticism}.
        \item To find at least one \textbf{objective moral value}, \textbf{Nagel uses "the view from nowhere"}.
        \item \textbf{The view from nowhere} involves \textbf{bracketing ourselves off} from the \textbf{world} and our \textbf{relationships} with people in the world. Essentially we stand back and look at our lives and the world we live in as a whole.
        \item According to Nagel, \textbf{the view from nowhere} abandons the \textbf{subjective viewpoint} leaving only an \textbf{objective viewpoint}.
        \item Nagel argues that \textbf{a maximally objective value} found by \textbf{increasing levels of abstraction} will have \textbf{three characteristics}:
        \begin{enumerate}
            \item The value will be \textbf{broad in scope}, it will apply to many different kinds of actions.
            \item The value will be \textbf{agent-neutral}, meaning it won't involve any reference to anyone.
            \item The value will be \textbf{external}, meaning that it won't refer to anyone's reasons for acting.
        \end{enumerate}
        \item Nagel warns about \textbf{the danger of taking the view from nowhere} is that if the value is \textbf{too abstract} is becomes \textbf{vacuous} (For example: Be good).
    \end{itemize}

    \section*{\centering{The Nature of Moral Explanation}}

    \subsection*{Gilbert Harman's Ethics and Observation}
    \begin{itemize}
        \item \textbf{Gilbert Harman} investiages the possibility that \textbf{ethics is objective like science}, given that both can be \textbf{judged} using \textbf{thought experiments}.
        \item If \textbf{ethics} is \textbf{analogous to science}, then just as \textbf{scientific theories} can be \textbf{supported (corroborated) by observation}, so can \textbf{ethical theories}.
        \item Harman contends that \textbf{scientific theories entail scientific law}. He then goes on to provide an example of an \textbf{observation} that is \textbf{NOT explained} by any \textbf{moral theory} (Someone observers kids burning a cat, and says it is morally wrong. The observation does not corroborate the theory that burning cats is morally wrong).
        \item Due to his counterexample, \textbf{observation does not corroborate theory} in ethics.
        \item In \textbf{science}, the \textbf{theory explains the observation}. In \textbf{ethics}, the \textbf{theory} does \textbf{not explain the oberservation}. Therefore \textbf{ethics} is \textbf{not a branch of science}. Hence \textbf{ethics is not analogous to science}.
        \item Harman then suggest that \textbf{ethics} is more like \textbf{mathematics} than it is like a \textbf{science}. \textbf{Mathematical statements do not explain observations} nor are they \textbf{corroborated by them}. Instead \textbf{theorems are derived from axioms, not observation}.
        \item Harman concludes that \textbf{ethics is not like mathematics} as \textbf{science uses mathematics}, thus \textbf{mathematics are indirectly corroborated by observation}.
        \item Since \textbf{ethics} is not \textbf{analogous to science or mathematics}, therefore \textbf{it is not objective}. Thus \textbf{ethics is subjective}.
    \end{itemize}

    \subsection*{Nicholas Sturgeon's Moral Explanations}
    \begin{itemize}
        \item \textbf{Nicholas Sturgeon} has an issue with \textbf{Harman's burning of the cat example}. He argues that indeed, \textbf{mainstream moral theories} do explain why the \textbf{kids burned the cat}, thus the \textbf{observation does corroborate the moral theory}.
        \item He claims that the theory that the \textbf{mainstream moral theories} entail that \textbf{anyone who burns a cat for fun} is suffering from \textbf{moral depravity}.
        \item Therefore \textbf{ethical theories} are \textbf{no worse} than \textbf{scientific theories}.
    \end{itemize}

    \section*{\centering{The Meaning of Moral Terms}}
    
    \subsection*{G.E. Moore}
    \begin{itemize}
        \item \textbf{G.E. Moore} comments extensively on \textbf{the meaning} of the term \textbf{"good"} in a famous work called the \textbf{Principia Ethica}.
        \item He argues against the tendency to \textbf{equate "good"} with a \textbf{set of natural properties} given such considerations as the following:
        \begin{enumerate}
            \item If we try to identify the term good with a natural property such as pleasure (like Mill did) then we have: Pleasure is good. But if good is nothing but pleasure, the statement "Good is pleasure" becomes "Pleasure is pleasure", which is a vacuous statement.
            \item The term "good" satisfies different criteria in different contexts.
        \end{enumerate}
        \item Moore claims that trying to \textbf{define "good"} in terms of \textbf{natural properties} commits the \textbf{"naturalistic fallacy"}; it is an \textbf{error} in \textbf{reasoning} that involves \textbf{inferring values from facts}.
        \item He claims that the same is true for \textbf{perceptual terms} (such as color). They \textbf{cannot be identified} with \textbf{natural properties}.
    \end{itemize}

    \subsection*{A.J. Ayer}
    \begin{itemize}
        \item \textbf{A.J. Ayer} claims that \textbf{statements of ethics, theology, and aesthetics} are all \textbf{meaningless} since they are \textbf{neither truths of logic} nor \textbf{empirically verifiable}. For example:
        \begin{enumerate}
            \item Theology | God exists.
            \item Ethics | Stealing is wrong.
            \item Aesthetics | This painting is beautiful.
        \end{enumerate}
    \end{itemize}

\end{document}