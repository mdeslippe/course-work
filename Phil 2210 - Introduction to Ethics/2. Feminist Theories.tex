\documentclass{article}
\usepackage[english]{babel}
\usepackage{longtable}
\usepackage[top=1in, bottom=0.25in, left=1.25in, right=1.25in,includefoot,heightrounded]{geometry}
\usepackage{indentfirst}
\usepackage[utf8]{inputenc}
\usepackage{amsmath,amssymb}
\usepackage{graphicx,tikz}
\usepackage{hyperref}
\usepackage[colorinlistoftodos]{todonotes}
\usepackage[document]{ragged2e}
\usepackage{fancyhdr}
\usepackage{enumerate}
\usepackage{listings}
\usepackage{color}
\usepackage{flowchart}
\usepackage{hyperref}
\usepackage{graphicx}
\usetikzlibrary{arrows}

\usetikzlibrary{shapes.geometric, arrows}
\tikzstyle{startstop} = [rectangle, rounded corners, minimum width=3cm, minimum height=1cm,text centered, draw=black, fill=red!30]
\tikzstyle{decision} = [diamond, minimum width=4cm, minimum height=0.5cm, text centered, draw=black, fill=green!30]
\tikzstyle{process} = [rectangle, minimum width=3cm, minimum height=1cm, text centered, draw=black, fill=orange!30]
\tikzstyle{arrow} = [thick,->,>=stealth]
\tikzstyle{io} = [trapezium, trapezium left angle=70, trapezium right angle=110, minimum width=2cm, text width=4cm, minimum height=1cm, text centered, draw=black, fill=blue!30]

\pagestyle{fancy}
\fancyhf{}
\lhead{Myles Deslippe}
\rhead{Phil 2210 | Introduction to Ethics}
\cfoot{\thepage}

\definecolor{MyDarkGreen}{rgb}{0.0,0.4,0.0}
\lstset{inputencoding=ansinew}
\lstset{breaklines=true} 

\begin{document}

    \section*{\centering{Moral Development}}

    \subsection*{Kohlberg's Theory of Moral Development}
    \begin{itemize}
        \item William Kohlberg wanted to extend Piaget's theory of development to moral development.
        \item According to Kohlberg, there are \textbf{six stages of moral development}:
        \begin{enumerate}
            \item The child is \textbf{entirely selfish}, and does not recognize the needs of others (psychological egosim).
            \item The child \textbf{learns} that by \textbf{begin kind} to others, there will be reciprociy (ethical egoism).
            \item The individual sees the \textbf{family} as the significant moral unit, such that what is morally good, is what is good for the family (virtue ethics).
            \item The individual sees the \textbf{nation and laws} as the significant moral unit, so that what is morally good, is good for the nation (social contract theory).
            \item The individual begins to \textbf{question all of their values}, abandoning both the family and the law (cultural relativism).
            \item The individual begins to \textbf{regard moral laws as being universally valid} (deontology). 
        \end{enumerate}
        \begin{itemize}
            \item The first two stages are said to be \textbf{pre-conventional stages of moral development}.
            \item The middle two stages are said to be \textbf{conventional stages of moral devlopment}.
            \item The last two stages are said to be \textbf{post-conventional stages of moral development}.
            \item Kohl regards \textbf{stage six} as being the \textbf{highest stage of moral development}.
            \item Kohl also claimed that \textbf{women} only make it to \textbf{stage 3} since they \textbf{tend to stay at home} (this was before the women's liberation, when men ran everything).
        \end{itemize}
    \end{itemize}

    \subsection*{Carol Gilligan's Critique of Kohlberg's Theory}
    \begin{itemize}
        \item Carol Gilligan argued that woman and ment simply \textbf{evaluate ethical dilemmas differently}. \textbf{Men} tend to view ethical dilemmas \textbf{legalistically} by appeal to universal moral rules, \textbf{women} tend to view ethical dilemmas \textbf{in terms of the concepts of care and concern}.
        \item She believes that \textbf{both ethical perspectives} have \textbf{something to offer}.
        \item She also argued that in \textbf{terms of ethical evaluations}, \textbf{women's opinions} are often \textbf{undervalued and ignored}.
        \item The following are some \textbf{dichotomies} between \textbf{male} and \textbf{female} approaches to ethics, alleged by Gilligan:
        \begin{enumerate}
            \item Men employ abstract moral principles in making moral evaluations, whereas women do not appeal to general principles.
            \begin{itemize}
                \item George Sher criticized this by saying that Women also employ moral rules in making decisions, but their rules are more nuanced than the rules employed by men.
            \end{itemize}
            \item Men make their moral decisions using abstract principles without reference to context, whereas women make their moral decisions by appealing to the contextual features of a given situation.
            \begin{itemize}
                \item George Sher criticized this by saying that even abstract rules cannot be applied unless the context of the situation is known.
            \end{itemize}
            \item Men rely on te concepts of justice and rules, whereas women rely on the concepts of care and concern.
            \begin{itemize}
                \item George Sher criticized this by saying t hat moral rules are not inconsistent with care and concern, and for that matter, care also involves consideration of quality, justice, and fairness.
            \end{itemize} 
        \end{enumerate}
    \end{itemize}

    \section*{The Ethics of Care}

    \subsection*{Virginia Held's Ethics of Care (Feminist Ethics)}
    \begin{itemize}
        \item Virginia Held argues that \textbf{the ethics of care} has the following important features:
        \begin{enumerate}
            \item The \textbf{central focus} of a \textbf{care ethic} is attending to the \textbf{needs of others}, especially with respect to whom we are \textbf{responsible for} (family, friends, etc).
            \item The \textbf{ethics of care} \textbf{condones} the use of \textbf{some emotions} in making \textbf{ethical decisions}, but not all emotions (such as envy, greed, etc).
            \item The \textbf{ethics of care rejects} the use of \textbf{abstract reasoning} in making \textbf{ethical decisions}.
            \item The \textbf{care of ethics} re-conceptualizes traditional notions about \textbf{the public and the private} (for example: women stay home in privagte, while men go out and work in public).
            \item The \textbf{ethics of care} uses the \textbf{concept of a person} as a \textbf{relation being} as opposed to being a \textbf{self-sufficient, stand-alone rational agent}.
        \end{enumerate}
        \item Held notes that \textbf{traditional moral theories} portray \textbf{moral dilemmas} as involving a \textbf{conflict} between \textbf{egoistic concerns} and \textbf{universal moral principles}. This is something that is \textbf{omitted} from \textbf{the ethics of care}.
        \item She also notes that \textbf{traditional moral theories} that ignore the \textbf{domain on the household and the family} leads to \textbf{unequal power dynamics} that \textbf{women and children do not choose}.
    \end{itemize}

    \subsection*{Marilyn Friedman's Beyond Caring}
    \begin{itemize}
        \item Friedman summarized Gilligan's view on women and ethics: 
        \begin{itemize}
            \item The \textbf{male moral voice} is given preference in a \textbf{patriarchial society}. In such a society, \textbf{moral judgements} are derived from \textbf{abstract universal moral rules}.
            \item The main concern of \textbf{male-based ethics} is with \textbf{rights and justice}.
            \item The \textbf{female moral voice}, which is \textbf{under-valued}, and largely \textbf{ignore}, derives \textbf{moral judgments} from \textbf{contextualize situations that are unique}.
            \item \textbf{Moral judgments} involve \textbf{empathy, compassion, care, concern} and the \textbf{avoidance of selfishness}.
        \end{itemize}
        \item Friedman's article has three main points:
        \begin{enumerate}
            \item Even if there is \textbf{not sufficient evidence }to confirm Gilligan's view that \textbf{women and men reason differently about morality}, there is a \textbf{culturally-based difference} between the \textbf{two genders}. Which are acquired through socialization.
            \item The concepts of \textbf{care and justice overlap}, which may explain the \textbf{lack of evidence} for Gilligan's claims.
            \item Even if \textbf{care and justice overlap}, \textbf{care} emphasizes \textbf{responsiveness to others} and their \textbf{uniqueness}, whereas \textbf{justice} emphasizes \textbf{adherence to abstract rules}.
        \end{enumerate}        
        \item Friedman distinguishes between two hypotheses advanced by Gilligan (she rejects the first one, and accepts the second one):
        \begin{enumerate}
            \item \textbf{The Different Voice Hypotheses} | The care perspective is distinct from the moral perspective which is centered on justice and rights.
            \item \textbf{The Gender Difference Hypothesis} | The care perspective is typically women's moral voice, whereas the justice perspective is typically men's moral voice.
        \end{enumerate}
        \item Her conclusion is that "We need nothing less thatn to 'demoralize' the genders, advance beyond the discussion of justice from care, and enlarge the symbolic access of each gender to all available conceptual and social resources for the sustenance and enrichment of our collective moral life".
    \end{itemize}

\end{document}