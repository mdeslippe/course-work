\documentclass[16pt]{article}
\usepackage[english]{babel}
\usepackage{longtable}
\usepackage[top=1in, bottom=0.25in, left=1.25in, right=1.25in,includefoot,heightrounded]{geometry}
\usepackage{indentfirst}
\usepackage[utf8]{inputenc}
\usepackage{amsmath,amssymb}
\usepackage{graphicx,tikz}
\usepackage{hyperref}
\usepackage[colorinlistoftodos]{todonotes}
\usepackage[document]{ragged2e}
\usepackage{fancyhdr}
\usepackage{enumerate}
\usepackage{listings}
\usepackage{color}
\usepackage{flowchart}
\usepackage{hyperref}
\usepackage{graphicx}
\usetikzlibrary{arrows}

\usetikzlibrary{shapes.geometric, arrows}
\tikzstyle{startstop} = [rectangle, rounded corners, minimum width=3cm, minimum height=1cm,text centered, draw=black, fill=red!30]
\tikzstyle{decision} = [diamond, minimum width=4cm, minimum height=0.5cm, text centered, draw=black, fill=green!30]
\tikzstyle{process} = [rectangle, minimum width=3cm, minimum height=1cm, text centered, draw=black, fill=orange!30]
\tikzstyle{arrow} = [thick,->,>=stealth]
\tikzstyle{io} = [trapezium, trapezium left angle=70, trapezium right angle=110, minimum width=2cm, text width=4cm, minimum height=1cm, text centered, draw=black, fill=blue!30]

\pagestyle{fancy}
\fancyhf{}
\lhead{Myles Deslippe}
\rhead{Phil 2210 | Introduction to Ethics}
\cfoot{\thepage}

\definecolor{MyDarkGreen}{rgb}{0.0,0.4,0.0}
\lstset{inputencoding=ansinew}
\lstset{breaklines=true} 

\begin{document}

    \section*{\centering{Morality and Religion}}

    \subsection*{Attempts to Ground Morality in Religion}
    \begin{itemize}
        \item There is a widely held view that there is an \textbf{intimate connection} between \textbf{morality and religion}.
        \begin{itemize}
            \item This implies that if there is no god, then there is no morality.
        \end{itemize}
        \item Even if there is a connection between \textbf{morality and religion}, attempts to \textbf{understand} this connection so far have \textbf{failed}.
        \item Two \textbf{popular} attempts to \textbf{understand the connection} between \textbf{morality and religion} are \textbf{Divine Command Theory}, and \textbf{Natural Law Theory.}
    \end{itemize}

    \subsection*{Divine Command Theory (DCT)}
    \begin{itemize}
        \item The \textbf{roots} of \textbf{DCT} have their origin in the writings of \textbf{Plato} in a dialogue called \textbf{The Euthyphro}.
        \item The basic ideas of \textbf{DCT} can be summarized as follows: 
        \begin{enumerate}
            \item An act is \textbf{obligatory} if God commands it (tells you to do it).
            \item An act is \textbf{permissible} if God permits it.
            \item An act is \textbf{forbidden} if God forbids it (tells you to not do it).
        \end{enumerate}
        \item This raises the question: "How do you know what God commands?"
        \begin{itemize}
            \item Some people believe that it is found in \textbf{religious scripture}.
        \end{itemize}
        \item An \textbf{ontological problem} (a problem in the study of existance) that relates to \textbf{DCT} is: how do we know that God exists?
        \begin{itemize}
            \item There is an ontological proof that goes as follows: God's essence is perfection. It is more perfect to exist than to not exist. Thus, God exists.
            \item However, not everyone accepts theological proofs of God's existance.
        \end{itemize}
        \item Some defenders of \textbf{DCT} believe that it is \textbf{not necessary} to \textbf{believe in God} to adhere to \textbf{DCT}. They would instead argue that \textbf{only if God exists}, morality is possible. So they are giving an account of what morality would be if there were a God.
        \item An \textbf{epistemological problem} (a problem about how we know something) with \textbf{DCT} is: even if God exists, how do we \textbf{know} what \textbf{God's commands are}?
        \begin{itemize}
            \item Some would argue that we know through \textbf{scripture, prayer, meditation, and religious experiences}.
            \item Adams claims that all \textbf{wrong actions} have one thing in common; they are \textbf{contrary} to the \textbf{commands} of a \textbf{loving God}.
        \end{itemize}
    \end{itemize}

    \subsection*{The Major Problem with DCT}
    \begin{itemize}
        \item The \textbf{major problem} with \textbf{DCT} was put forward by \textbf{Plato} in \textbf{Euthyphro}. The \textbf{problem} is known as the \textbf{Euthyprho dilemma}.
        \item The \textbf{Euthyprho dilemma is as follows}: Does god command an action because it is right, or is the action right because God commands it?
        \begin{itemize}
            \item Suppose an action X is right, simply because God commands it. Doesn't that make morality arbitrary? It can be whatver God wants? He can change the commands at any time. God could command us to do something that we think is immoral.
            \item Suppose an action X is right, then God will command it. But this means, that religion is based on morality instead of morality being based on religion. God is essentially acting as a messanger for something above him.
        \end{itemize}
        \item This dilemma leads us to \textbf{abandon DCT}. Though there have been attempts to resuscitate it.
        \begin{itemize}
            \item One theory that attempted to save DCT is \textbf{A New Device Command Theory} by \textbf{Robert Adams}. To get around the \textbf{Euthrphro dilemma} it modifies \textbf{DCT} as follows: An act is morally right if is commanded by a \textbf{loving God}; a loving god would never command us to do something that is morally wrong by our moral intuitions.
        \end{itemize}
    \end{itemize}    

    \subsection*{Natural Law Theory (NLT)}
    \begin{itemize}
        \item 
    \end{itemize}

\end{document}