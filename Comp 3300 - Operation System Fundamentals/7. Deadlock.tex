\documentclass{article}
\usepackage[english]{babel}
\usepackage{longtable}
\usepackage[top=1in, bottom=0.25in, left=1.25in, right=1.25in,includefoot,heightrounded]{geometry}
\usepackage{indentfirst}
\usepackage[utf8]{inputenc}
\usepackage{amsmath,amssymb}
\usepackage{graphicx,tikz}
\usepackage{hyperref}
\usepackage[colorinlistoftodos]{todonotes}
\usepackage[document]{ragged2e}
\usepackage{fancyhdr}
\usepackage{enumerate}
\usepackage{listings}
\usepackage{color}
\usepackage{flowchart}
\usepackage{hyperref}
\usetikzlibrary{arrows}

\usetikzlibrary{shapes.geometric, arrows}
\tikzstyle{startstop} = [rectangle, rounded corners, minimum width=3cm, minimum height=1cm,text centered, draw=black, fill=red!30]
\tikzstyle{decision} = [diamond, minimum width=4cm, minimum height=0.5cm, text centered, draw=black, fill=green!30]
\tikzstyle{process} = [rectangle, minimum width=3cm, minimum height=1cm, text centered, draw=black, fill=orange!30]
\tikzstyle{arrow} = [thick,->,>=stealth]
\tikzstyle{io} = [trapezium, trapezium left angle=70, trapezium right angle=110, minimum width=2cm, text width=4cm, minimum height=1cm, text centered, draw=black, fill=blue!30]

\pagestyle{fancy}
\fancyhf{}
\lhead{Myles Deslippe}
\rhead{Comp 3300 | Operating System Fundamentals}
\cfoot{\thepage}

\definecolor{MyDarkGreen}{rgb}{0.0,0.4,0.0}
\lstset{inputencoding=ansinew}
\lstset{breaklines=true} 

\begin{document}

    \section*{\centering{Deadlock}}

    \subsection*{System Model}
    \begin{itemize}
        \item \textbf{Computer systems} consist of \textbf{resources} (for example CPU, memory, and I/O devices).
        \item We also consider tools like \textbf{locks, and semaphores} to be \textbf{resources}.
        \item Each resource of type $R_i$ has $W_i$ instacnes.
        \item When a \textbf{process} utilizes a resource the following happens:
        \begin{enumerate}
            \item The process \textbf{requests} the resource.
            \item The process \textbf{uses} the resource.
            \item The process \textbf{releases} the resource.
        \end{enumerate}
        \begin{itemize}
            \item Processes can request as many resources as they need, it is not restricted. However, the number of resources requested cannot exceed the amount of resources available.
            \item Processes may need to wait for resources to become available before their request is granted.
        \end{itemize}
        \item The operating system uses a \textbf{system table} to record the status of \textbf{resources} (unallocated / allocated). If a resource is allocated, the process / thread allocating it is also recorded.
    \end{itemize}

    \subsection*{Deadlock}
    \begin{itemize}
        \item A set of \textbf{threads} are said to be \textbf{in a deadlocked state} when \textbf{every thred} in the set is \textbf{waiting for an event} that can be caused only by \textbf{another thread in the set}.
        \item The following is an example of how deadlock can occur:
        \begin{enumerate}
            \item Process one acquires a lock on resource 1.
            \item Process two acquires a lock on resource 2.
            \item Process wants to acquire a lock on resource two, but must block until it is available.
            \item Process two wants to acquire a lock on resource one, but must block until it is available.
            \item Both processes have entered deadlock.
        \end{enumerate}
        \item A set of \textbf{threads} are said to be \textbf{in a livelocked state} when \textbf{every thread} in the set is \textbf{waiting for an event} that can be caused only by \textbf{another thread in the set}. Livelock occurs when a the thread that the others are waiting for \textbf{continuously attempts an action that fails}.
    \end{itemize}

    \subsection*{Deadlock Conditions}
    \begin{itemize}
        \item \textbf{Deadlock} can arise if \textbf{four conditions} occur at the same time:
        \begin{enumerate}
            \item \textbf{Mutual exclusion}: Only one process at a time an use the resource.
            \item \textbf{Hold and wait}: A process holding at least one resource is waiting to acquire additional resources held by other processes.
            \item \textbf{No preemption}: A resource can be released only voluntarily by the process holding it.
            \item \textbf{Circular wait}: There exists a set of waiting processes such that $P_0$ is waiting for $P_1$, and $P_1$ is waiting for $P_2$ and ... and $P_n$ is waiting for $P_0$.
        \end{enumerate}
        \item These conditions do \textbf{not guarantee deadlock} will occur, they can \textbf{potentially cause deadlock}.
    \end{itemize}

    \subsection*{Resource-Allocation Graphs}
    \begin{itemize}
        \item A \textbf{resource-allocation graph} is a \textbf{directed graph}, $G=(V,E)$.
        \item $V$ is partitioned into two sets:
        \begin{enumerate}
            \item $P=\{P_1, P_2, ..., P_n\}$ consisting of all of the processes in the system.
            \item $R=\{R_1, R_2, ..., R_m\}$ consisting of all of the resource types in the system.
        \end{enumerate}
        \item $E$ consists of two types of edges:
        \begin{enumerate}
            \item A \textbf{request edge} is a directed edge of the form $P_i \rightarrow R_j$.
            \item An \textbf{assignment edge} is a directed edge of the form $R_j \rightarrow P_i$.
        \end{enumerate}
        \item If $G$ has a \textbf{cycle}, \textbf{deadlock} may exist.
        \item If $G$ has \textbf{no cycles}, \textbf{deadlock} does not exist.
    \end{itemize}

    \subsection*{Methods for Handling Deadlock}
    \begin{itemize}
        \item One way to handle deadlocks is to ensure the system will never enter a deadlocked state (prevention / avoidence).
        \item Another way to handle deadlocks is to allow the system to enter a deadlocked state, and then recover.
        \item The last way is to ignore the problem, and pretend that deadlocks never occur in the system (this option is used by most systems).
    \end{itemize}

\end{document}
