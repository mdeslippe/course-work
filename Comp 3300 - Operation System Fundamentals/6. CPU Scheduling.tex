\documentclass{article}
\usepackage[english]{babel}
\usepackage{longtable}
\usepackage[top=1in, bottom=0.25in, left=1.25in, right=1.25in,includefoot,heightrounded]{geometry}
\usepackage{indentfirst}
\usepackage[utf8]{inputenc}
\usepackage{amsmath,amssymb}
\usepackage{graphicx,tikz}
\usepackage{hyperref}
\usepackage[colorinlistoftodos]{todonotes}
\usepackage[document]{ragged2e}
\usepackage{fancyhdr}
\usepackage{enumerate}
\usepackage{listings}
\usepackage{color}
\usepackage{flowchart}
\usepackage{hyperref}
\usetikzlibrary{arrows}

\usetikzlibrary{shapes.geometric, arrows}
\tikzstyle{startstop} = [rectangle, rounded corners, minimum width=3cm, minimum height=1cm,text centered, draw=black, fill=red!30]
\tikzstyle{decision} = [diamond, minimum width=4cm, minimum height=0.5cm, text centered, draw=black, fill=green!30]
\tikzstyle{process} = [rectangle, minimum width=3cm, minimum height=1cm, text centered, draw=black, fill=orange!30]
\tikzstyle{arrow} = [thick,->,>=stealth]
\tikzstyle{io} = [trapezium, trapezium left angle=70, trapezium right angle=110, minimum width=2cm, text width=4cm, minimum height=1cm, text centered, draw=black, fill=blue!30]

\pagestyle{fancy}
\fancyhf{}
\lhead{Myles Deslippe}
\rhead{Comp 3300 | Operating System Fundamentals}
\cfoot{\thepage}

\definecolor{MyDarkGreen}{rgb}{0.0,0.4,0.0}
\lstset{inputencoding=ansinew}
\lstset{breaklines=true} 

\begin{document}

\section*{\centering{CPU Scheduling}}

    \subsection*{CPU Scheduling}
    \begin{itemize}
        \item \textbf{CPU scheduling} is the task performed by the \textbf{CPU} that decides \textbf{the order processes should be executed}.
        \item To \textbf{schedule processes} a \textbf{queue} is used.
    \end{itemize}

    \subsection*{Types of Processes}
    \begin{itemize}
        \item There are \textbf{two types} of \textbf{processes}:
        \begin{enumerate}
            \item \textbf{I/O Bound} processes are processes that have small bursts of CPU activity, and then wait for I/O. These type of processes directly affect the user interaction, so they should have a higher priority.
            \item \textbf{CPU Bound} processes are processes that have little to no I/O, they mostly perform computations. These types of processes are able to function with a lower priority.
        \end{enumerate}
    \end{itemize}

    \subsection*{Scheduling Criteria}
    \begin{itemize}
        \item \textbf{Maximize CPU utilization} | We want to keep the CPU as busy as possible.
        \item \textbf{Maximize throughput} | We want to maximize the amount of processes that complete their execution per time unit.
        \item \textbf{Minimize turnaround time} | We want to minimize the amount of time it takes to execute a process.
        \item \textbf{Minimize waiting time} | We want to minimize the amount of time a processes has to wait before getting executed.
        \item \textbf{Minimize response time} | We want to minimize the amount of time it takes from when a request was submitted, to the first response.
        \item \textbf{Fairness} | We want to give each process a fair share of the CPU.
    \end{itemize}

    \subsection*{First Come First Server (FCFS) Scheduling}
    \begin{itemize}
        \item This algorithm works by \textbf{fully executing processes} in the order they are \textbf{placed into the queue}.
        \item This algorithm is simple, and fair. However, it wait time depends on the arrival time, and short processes will be stuck waiting for long processes to complete.
        \item This algorithm could also cause the system to half if a single process enters an infinite loop.
    \end{itemize}

    \subsection*{Shortest-Job-First (SJF) Scheduling}
    \begin{itemize}
        \item This algorithm works by \textbf{associating} each \textbf{processes} by the length of it's \textbf{next CPU burst}. These lengths are then used to schedule the processes with the shortest time first.
        \item Implementing this algorithm with \textbf{no preemption}, the process continues to execute until it's CPU burst is complete.
        \item Implementing this algorithm with \textbf{preemption}, the process continues to execute but may be paused to switch to a shorter process that enteres the queue.
        \item This algorithm has a more optimal minimum average time, but is not pratical. It is too difficult to predict burt times, and may lead to starvation of really long jobs.
    \end{itemize}

    \subsection*{Round Robin Scheduling}
    \begin{itemize}
        \item This algorithm works by assigning \textbf{each process} a \textbf{small unit of CPU time} (time quantum q). This time is ussually 10-100 milliseconds. \textbf{After this time has elapsed}, the \textbf{process} is \textbf{preempted} and added to the \textbf{end of the ready queue}.
    \end{itemize}

    \subsection*{Priority Scheduling}
    \begin{itemize}
        \item A \textbf{priority} is an \textbf{integer} associated with \textbf{each process}.
        \item In this algorithm, the \textbf{CPU} is allocated to the \textbf{process with the highest priority} (smallest integer).
        \begin{itemize}
            \item There is a \textbf{preemtive}, and \textbf{non-preemtive} version of the algorithm.
        \end{itemize}
        \item \textbf{SFJ} is \textbf{priority scheduling} where priority is the inverse of predicted next CPU burst time.
        \item The main problem with this algorithm is that \textbf{low priority processes} may \textbf{never execute}. The solotuon to this is \textbf{aging}, as a \textbf{processes ages, it's priority is increased}.
    \end{itemize}

    \subsection*{Multi-Level Queue Scheduling}
    \begin{itemize}
        \item \textbf{Multi-Level queue scheduling} works by creating a ready queue, and \textbf{partitioning it into seperate queues} (eg foreground, background).
        \item \textbf{Processes} permanently stay in \textbf{a single partition}.
        \item Each \textbf{queue} has it's own \textbf{scheduling algorithm} (foreground uses RR, background uses FCFS).
        \item \textbf{Scheduling} must be done \textbf{between the queues}. To do this, \textbf{time slicing} is used. Each queue gets a \textbf{certain amount of CPU time}, which it can schedule amongst \textbf{it's processees}.
    \end{itemize}

    \subsection*{Multi-Level Feedback Queue Scheduling}
    \begin{itemize}
        \item The \textbf{Multi-Level queue} algorithm does not allow process to \textbf{change queues}, which is a problem, as the functionality of \textbf{processes can change over time}. (eg from foreground to background).
        \item The \textbf{Multi-Level feedback queue} algorithm allows processes to \textbf{change queues} based on the \textbf{characteristics of their CPU bursts}.
    \end{itemize}

\end{document}
