\documentclass[16pt]{article}
\usepackage[english]{babel}
\usepackage{longtable}
\usepackage[top=1in, bottom=1.25in, left=1.25in, right=1.25in,includefoot,heightrounded]{geometry}
\usepackage{indentfirst}
\usepackage[utf8]{inputenc}
\usepackage{amsmath,amssymb}
\usepackage{graphicx,tikz}
\usepackage{hyperref}
\usepackage[colorinlistoftodos]{todonotes}
\usepackage[document]{ragged2e}
\usepackage{fancyhdr}
\usepackage{enumerate}
\usepackage{listings}
\usepackage{color}
\usepackage{flowchart}
\usepackage{hyperref}
\usetikzlibrary{arrows}


\usetikzlibrary{shapes.geometric, arrows}
\tikzstyle{startstop} = [rectangle, rounded corners, minimum width=3cm, minimum height=1cm,text centered, draw=black, fill=red!30]
\tikzstyle{decision} = [diamond, minimum width=4cm, minimum height=0.5cm, text centered, draw=black, fill=green!30]
\tikzstyle{process} = [rectangle, minimum width=3cm, minimum height=1cm, text centered, draw=black, fill=orange!30]
\tikzstyle{arrow} = [thick,->,>=stealth]
\tikzstyle{io} = [trapezium, trapezium left angle=70, trapezium right angle=110, minimum width=2cm, text width=4cm, minimum height=1cm, text centered, draw=black, fill=blue!30]

\pagestyle{fancy}
\fancyhf{}
\lhead{Myles Deslippe}
\rhead{Comp 3300 | Operating System Fundamentals}
\cfoot{\thepage}

\renewcommand{\headrulewidth}{0.4pt}
\renewcommand{\baselinestretch}{1.2}
\renewcommand{\labelitemi}{$\circ$}

\definecolor{MyDarkGreen}{rgb}{0.0,0.4,0.0}
\lstset{inputencoding=ansinew}
\lstset{breaklines=true} 
\begin{document}

\section*{\centering{Introduction}}

    \subsection*{Operating Systems}
    \begin{itemize}
        \item An \textbf{Operating System} is system software that manages computer hardware, software resources, and provides common services to all computer programs.
        \item Operating Systems provide a layer of \textbf{abstraction} that programs can use to perform operations that are \textbf{independent} of the physical hardware.
        \item There are \textbf{two main goals} for any operating system:
        \begin{enumerate}
            \item Provide a \textbf{user-friendly environment}, that allows the user to execute their desired programs.
            \item Manage system resources as \textbf{efficiently} as possible.
        \end{enumerate}
    \end{itemize}
    
    \subsection*{Types of Operating Systems}
    \begin{itemize}
        \item There are many types of operating systems, ranging from \textbf{general-purpose operating systems}, to \textbf{embedded operating systems}.
        \item A \textbf{general-purpose operating system} is an operating system that support process management, memory management, IO devices, a file system, and a user interface. That can solve a wide range of problems.
        \item An \textbf{embedded operating system} is a specialized operating system designed to perform a specific task for a specific device. They often lack many features that general-purpose operating systems have. 
        \item There are two ways operating systems can be viewed:
        \begin{enumerate}
            \item \textbf{The User View} is concerned with what the end user will be using the operating system for.
            \item \textbf{The System View} is concerned with the way the operating system will control programs, and manage resources.
        \end{enumerate}
    \end{itemize}
    
    \subsection*{The Operating System Kernel}
    \begin{itemize}
        \item The \textbf{kernel is the core of an operating system}, and is a process that is \textbf{always running} when the system is on.
        \item The kernel \textbf{facilitates interactions} between \textbf{hardware components} and \textbf{software applications}.
    \end{itemize}

    \subsection*{Hardware Controllers}
    \begin{itemize}
        \item The \textbf{physical hardware components} of a computer system are managed by a \textbf{controller} which acts as an \textbf{intermediary} between the device and the rest of the system. Device controllers work by handling raw signals coming from the CPU and directing the hardware accordingly.
        \item The \textbf{controllers} are connected to the \textbf{system bus}, which gives the controllers access to \textbf{shared memory} that can be used to communicate with other components.
        \item \textbf{Drivers} are a special type of software that \textbf{manage devices}.
        \item To sum it up, \textbf{controllers} handle signals from the CPU and access shared memory. Whereas \textbf{drivers} are responsible for managing the device.
    \end{itemize}

    \subsection*{System Events}
    \begin{itemize}
        \item An \textbf{event} is an \textbf{action} or \textbf{occurrence} recognized by software.
        \item Operation system are \textbf{event driven}.
        \item There are three main categories of events:
        \begin{enumerate}
            \item \textbf{Hardware Interrupts} are events that are raised by \textbf{hardware devices}. They can occur at any time.
            \item \textbf{Software Interrupts (Traps)} are events that are raised by \textbf{programs to invoke an operating system functionality}.
            \item \textbf{Exceptions} are events that are \textbf{generated automatically} by the processor as the result of an \textbf{illegal instruction / operation}.
        \end{enumerate}
        \item There are two types of exception events:
        \begin{enumerate}
            \item \textbf{Faults} are \textbf{exceptions} that the program \textbf{can 
            recover from}.
            \item \textbf{Aborts} are \textbf{exceptions} that the program \textbf{cannot, or are very difficult to recover from}.
        \end{enumerate}
    \end{itemize}

\end{document}