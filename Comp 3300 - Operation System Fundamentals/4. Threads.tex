\documentclass{article}
\usepackage[english]{babel}
\usepackage{longtable}
\usepackage[top=1in, bottom=0.25in, left=1.25in, right=1.25in,includefoot,heightrounded]{geometry}
\usepackage{indentfirst}
\usepackage[utf8]{inputenc}
\usepackage{amsmath,amssymb}
\usepackage{graphicx,tikz}
\usepackage{hyperref}
\usepackage[colorinlistoftodos]{todonotes}
\usepackage[document]{ragged2e}
\usepackage{fancyhdr}
\usepackage{enumerate}
\usepackage{listings}
\usepackage{color}
\usepackage{flowchart}
\usepackage{hyperref}
\usetikzlibrary{arrows}

\usetikzlibrary{shapes.geometric, arrows}
\tikzstyle{startstop} = [rectangle, rounded corners, minimum width=3cm, minimum height=1cm,text centered, draw=black, fill=red!30]
\tikzstyle{decision} = [diamond, minimum width=4cm, minimum height=0.5cm, text centered, draw=black, fill=green!30]
\tikzstyle{process} = [rectangle, minimum width=3cm, minimum height=1cm, text centered, draw=black, fill=orange!30]
\tikzstyle{arrow} = [thick,->,>=stealth]
\tikzstyle{io} = [trapezium, trapezium left angle=70, trapezium right angle=110, minimum width=2cm, text width=4cm, minimum height=1cm, text centered, draw=black, fill=blue!30]

\pagestyle{fancy}
\fancyhf{}
\lhead{Myles Deslippe}
\rhead{Comp 3300 | Operating System Fundamentals}
\cfoot{\thepage}

\definecolor{MyDarkGreen}{rgb}{0.0,0.4,0.0}
\lstset{inputencoding=ansinew}
\lstset{breaklines=true} 

\begin{document}

\section*{\centering{Threads}}

\subsection*{Threads}
\begin{itemize}
    \item A \textbf{thread} is a \textbf{single sequential flow of control within a program}.
    \item \textbf{Threads} are the \textbf{smallest unit of processing} that can be performed in an \textbf{operating system}.
    \item A \textbf{single process} can have \textbf{more than one thread}; each thread having a unique \textbf{program counter, stack, thread id, and set of registers}.
    \item \textbf{All threads} running under a \textbf{single} process have \textbf{shared memory (instructions, data, heap)}. They also can \textbf{communicate with each other directly}; there is no need for IPC.
    \item The \textbf{amount of time} it takes to \textbf{switch between threads} is less than the \textbf{amount of time} it takes to \textbf{switch between processes}.
    \item On \textbf{computer systems} that have \textbf{multiple cores}, \textbf{threads can run in parallel}.
    \item[] \begin{center}
                \includegraphics*[width=\textwidth-25pt]{images/Threads.jpg}
            \end{center} 
\end{itemize}

\subsection*{Threads vs Processes}
\begin{itemize}
    \item \textbf{Threads} have \textbf{no data segment or heap}.
    \item A \textbf{process has code, data, heap, and stack segmetns}.
    \item A \textbf{thread} \textbf{cannot live on it's own}; it needs to be attached to a process.
    \item \textbf{Processes} must have \textbf{atleast one thread}.
    \item There can be \textbf{more than one thread in a single process}.
    \item \textbf{Threads} in a \textbf{process} share data.
    \item If a \textbf{thread dies}, it's \textbf{stack is reclaimed}.
    \item If a \textbf{process dies}, all of it's \textbf{threads die}.
\end{itemize}

\subsection*{POSIX Threads}
\begin{itemize}
    \item To \textbf{create a thread} you use the \textbf{$\text{pcreate\_thread(thread, attr, start\_routine, arg)}$ function}.
    \begin{itemize}
        \item \textbf{thread} | An opaque, unique identifier for the new thread returned by the subroutine.
        \item \textbf{attr} | An opaque attribute object that may be used to set thread attributes.
        \item \textbf{$\text{start\_routine}$} | The routine that the thread will execute one it has been created.
        \item \textbf{arg} | A single argument that may be passed to the \textbf{$\text{start\_routine}$}.
    \end{itemize}
    \item To \textbf{destroy a thread} you use the \textbf{$\text{pthread\_exit(status)}$ function}. This will \textbf{terminate the calling thread}, and \textbf{make the status available} to any \textbf{successful join with the terminating thread}.
    \item To \textbf{wait for a thread to terminate}, you use the \textbf{$\text{pthread\_join(threadid, status)}$ function}. This will \textbf{suspend the execution} of the \textbf{calling thread} until the \textbf{target thread terminates}.
\end{itemize}

\subsection*{User Threads and Kernel Threads}
\begin{itemize}
    \item \textbf{User threads} are \textbf{threads} that are \textbf{managed} at the \textbf{user-level}.
    \item \textbf{Kernel threads} are \textbf{threads} that are \textbf{managed by the kernel}.
    \item \textbf{User threads} and \textbf{kernel threads} have the same capabilities. 
\end{itemize}

\subsection*{Multithreading Models}
\begin{itemize}
    \item There are three multithreading models:
    \begin{enumerate}
        \item \textbf{Many-to-One}.
        \item \textbf{One-to-One}.
        \item \textbf{Many-to-Many}.
    \end{enumerate}
\end{itemize}

\end{document}
