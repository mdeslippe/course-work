\documentclass{article}
\usepackage[english]{babel}
\usepackage{longtable}
\usepackage[top=1in, bottom=0.25in, left=1.25in, right=1.25in,includefoot,heightrounded]{geometry}
\usepackage{indentfirst}
\usepackage[utf8]{inputenc}
\usepackage{amsmath,amssymb}
\usepackage{graphicx,tikz}
\usepackage{hyperref}
\usepackage[colorinlistoftodos]{todonotes}
\usepackage[document]{ragged2e}
\usepackage{fancyhdr}
\usepackage{enumerate}
\usepackage{listings}
\usepackage{color}
\usepackage{flowchart}
\usepackage{hyperref}
\usetikzlibrary{arrows}

\usetikzlibrary{shapes.geometric, arrows}
\tikzstyle{startstop} = [rectangle, rounded corners, minimum width=3cm, minimum height=1cm,text centered, draw=black, fill=red!30]
\tikzstyle{decision} = [diamond, minimum width=4cm, minimum height=0.5cm, text centered, draw=black, fill=green!30]
\tikzstyle{process} = [rectangle, minimum width=3cm, minimum height=1cm, text centered, draw=black, fill=orange!30]
\tikzstyle{arrow} = [thick,->,>=stealth]
\tikzstyle{io} = [trapezium, trapezium left angle=70, trapezium right angle=110, minimum width=2cm, text width=4cm, minimum height=1cm, text centered, draw=black, fill=blue!30]

\pagestyle{fancy}
\fancyhf{}
\lhead{Myles Deslippe}
\rhead{Comp 3300 | Operating System Fundamentals}
\cfoot{\thepage}

\definecolor{MyDarkGreen}{rgb}{0.0,0.4,0.0}
\lstset{inputencoding=ansinew}
\lstset{breaklines=true} 

\begin{document}

\section*{\centering{Software Synchronization}}

\subsection*{Race Conditions}
\begin{itemize}
    \item A \textbf{race condition} is a situation where \textbf{several processes or threads} attempt to \textbf{manipulate the same data}, and the \textbf{outcome} of the execution \textbf{depends on the particular order the access takes place}.
    \item The \textbf{section of code} where a \textbf{process or thread} accesses a \textbf{shared resource} is referred to as the \textbf{critical section}.
    \item To \textbf{prevent race conditions}, we use \textbf{synchronization}. This ensures that no more than \textbf{one process} is able to execute it's \textbf{critical section} at a time.
\end{itemize}

\subsection*{Syncronization Solution Requirements}
\begin{itemize}
    \item Any \textbf{solution} to \textbf{synchronization} should satisfy the following requirements:
    \begin{enumerate}
        \item The solution should have \textbf{mutual exclusion (mutex)}: Only \textbf{one process} can execute it's \textbf{critical section} at any given time.
        \item The solution should have \textbf{progress}: When \textbf{no process} is currently in a \textbf{critical section}, \textbf{any process} that \textbf{requests entry} into the \textbf{critical section} must be \textbf{permitted without delay}.
        \item The solution must \textbf{prevent starvation (bounded wait)}: There is an \textbf{upper bound} on the number of times a \textbf{process enters the critical section} while \textbf{another is waiting}.
    \end{enumerate}
    \begin{itemize}
        \item Such a solution will prevent race conditions.
    \end{itemize}
\end{itemize}

\subsection*{Synchronization Solutions}
\begin{itemize}
    \item A \textbf{simple} way to \textbf{achieve synchronization} is the use interrupts; when \textbf{interrupts are disabled, context switches will not happen}.
    \begin{itemize}
        \item This is not a good solution, becuase user processes generally cannot disable interrupts.
        \item This also does not work on a multi-core system.
    \end{itemize}
    \item Another way to \textbf{achieve synchronization} is by using a \textbf{variable as a flag} to \textbf{determine} if any other processes are executing their \textbf{critical section}. When a process wants to \textbf{execute their critical section} and another process has already locked the resource, the process will use a \textbf{conditional loop until the lock is removed}.
    \begin{itemize}
        \item This acheives mutual exclusion, but wastes a lot of processor time.
        \item This also does not prevent starvation.
    \end{itemize}
    \item \textbf{Modern computers} use \textbf{hardware solutions} to \textbf{synchronize processes}.
\end{itemize}

\section*{\centering{Hardware Syncronization}}

\end{document}
