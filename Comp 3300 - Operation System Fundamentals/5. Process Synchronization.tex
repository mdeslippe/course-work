\documentclass{article}
\usepackage[english]{babel}
\usepackage{longtable}
\usepackage[top=1in, bottom=0.25in, left=1.25in, right=1.25in,includefoot,heightrounded]{geometry}
\usepackage{indentfirst}
\usepackage[utf8]{inputenc}
\usepackage{amsmath,amssymb}
\usepackage{graphicx,tikz}
\usepackage{hyperref}
\usepackage[colorinlistoftodos]{todonotes}
\usepackage[document]{ragged2e}
\usepackage{fancyhdr}
\usepackage{enumerate}
\usepackage{listings}
\usepackage{color}
\usepackage{flowchart}
\usepackage{hyperref}
\usetikzlibrary{arrows}

\usetikzlibrary{shapes.geometric, arrows}
\tikzstyle{startstop} = [rectangle, rounded corners, minimum width=3cm, minimum height=1cm,text centered, draw=black, fill=red!30]
\tikzstyle{decision} = [diamond, minimum width=4cm, minimum height=0.5cm, text centered, draw=black, fill=green!30]
\tikzstyle{process} = [rectangle, minimum width=3cm, minimum height=1cm, text centered, draw=black, fill=orange!30]
\tikzstyle{arrow} = [thick,->,>=stealth]
\tikzstyle{io} = [trapezium, trapezium left angle=70, trapezium right angle=110, minimum width=2cm, text width=4cm, minimum height=1cm, text centered, draw=black, fill=blue!30]

\pagestyle{fancy}
\fancyhf{}
\lhead{Myles Deslippe}
\rhead{Comp 3300 | Operating System Fundamentals}
\cfoot{\thepage}

\definecolor{MyDarkGreen}{rgb}{0.0,0.4,0.0}
\lstset{inputencoding=ansinew}
\lstset{breaklines=true} 

\begin{document}

\section*{\centering{Software Synchronization}}

\subsection*{Race Conditions}
\begin{itemize}
    \item A \textbf{race condition} is a situation where \textbf{several processes or threads} attempt to \textbf{manipulate the same data}, and the \textbf{outcome} of the execution \textbf{depends on the particular order the access takes place}.
    \item The \textbf{section of code} where a \textbf{process or thread} accesses a \textbf{shared resource} is referred to as the \textbf{critical section}.
    \item To \textbf{prevent race conditions}, we use \textbf{synchronization}. This ensures that no more than \textbf{one process} is able to execute it's \textbf{critical section} at a time.
\end{itemize}

\subsection*{Syncronization Solution Requirements}
\begin{itemize}
    \item Any \textbf{solution} to \textbf{synchronization} should satisfy the following requirements:
    \begin{enumerate}
        \item The solution should have \textbf{mutual exclusion (mutex)}: Only \textbf{one process} can execute it's \textbf{critical section} at any given time.
        \item The solution should have \textbf{progress}: When \textbf{no process} is currently in a \textbf{critical section}, \textbf{any process} that \textbf{requests entry} into the \textbf{critical section} must be \textbf{permitted without delay}.
        \item The solution must \textbf{prevent starvation (bounded wait)}: There is an \textbf{upper bound} on the number of times a \textbf{process enters the critical section} while \textbf{another is waiting}.
    \end{enumerate}
    \begin{itemize}
        \item Such a solution will prevent race conditions.
    \end{itemize}
\end{itemize}

\subsection*{Synchronization Solutions}
\begin{itemize}
    \item A \textbf{simple} way to \textbf{achieve synchronization} is the use interrupts; when \textbf{interrupts are disabled, context switches will not happen}.
    \begin{itemize}
        \item This is not a good solution, becuase user processes generally cannot disable interrupts.
        \item This also does not work on a multi-core system.
    \end{itemize}
    \item Another way to \textbf{achieve synchronization} is by using a \textbf{variable as a flag} to \textbf{determine} if any other processes are executing their \textbf{critical section}. When a process wants to \textbf{execute their critical section} and another process has already locked the resource, the process will use a \textbf{conditional loop until the lock is removed}.
    \begin{itemize}
        \item This acheives mutual exclusion, but wastes a lot of processor time.
        \item This also does not prevent starvation.
    \end{itemize}
    \item \textbf{Modern computers} use \textbf{hardware solutions} to \textbf{synchronize processes}.
\end{itemize}

\section*{\centering{Hardware Syncronization}}

\subsubsection*{Hardware Solution}
\begin{itemize}
    \item The \textbf{hardware solution} always executes the \textbf{lock check, and set} in the same intruction. This avoids a \textbf{context-switch} inbetwwen, which would not allow for mutual exclusion.
    \item This solution is called a \textbf{Test and Set Intruction}. It works by writing to a memory location, and returning its old value.
    \begin{verbatim}
        // Instruction provided by hardware, 
        //not program code.
        // You use the return value to determine 
        //if the lock is set (0 => lock is unset, 1 => lock is set).
        int test_and_set(int *L) {
            int prev = *L;
            *L = 1;
            return prev;
        }    
    \end{verbatim}
    \item If \textbf{two processors} execute the \textbf{test-and-set instruction} at the same time, the \textbf{hardware ensures} that one \textbf{test-and-set} does \textbf{both} of it's steps before the other one starts.
\end{itemize}

\subsection*{Intel HD Support (xchg Instruction)}
\begin{itemize}
    \item \textbf{Intel CPUs} have an \textbf{xchg instruction}. If two CPUs execute \textbf{xchg} at the same time, the hardware ensures that only one \textbf{xchg completes}, before the second begins.
    \item The \textbf{xchg} instruction takes in two registers as \textbf{two operands} and \textbf{swaps their values} atomically. This behaviour acts identically to the test-and-set solution.
\end{itemize}

\subsection*{Hgh Level Constructs}
\begin{itemize}
    \item A \textbf{Spinlock} is a mechanism for locking a resource. It works as follows: One process will \textbf{acquire the lock}, adn the other processes will wait in a loop repeatedly checking if the lock is available, once the first process \textbf{releases the lock}, other processes will \textbf{acqire it}.
    \begin{verbatim}
        void acquire(int *locked) {
            int val = 1;
            while (1) {
                xchg(locked, &val)

                if(val == 0)
                    break;
            }
        }

        void release(int *locked) {
            *locked = 0;
        }
    \end{verbatim}
    \begin{itemize}
        \item This mechanism is inefficient, becuase the CPU will waste time executing the loops while other processes wait for the lock to be released.
        \item It is useful for short critical sections, but not when the period of wait in unpredicatable or will take a long time.
    \end{itemize}
    \item A \textbf{Mutex} is a mechanism for locking a resource, if the lock is set, other processes that need accesses to the lock resource will \textbf{sleep} until it is release. When the processes that has \textbf{currently acquired the lock releases it}, it will \textbf{wake up the other sleeping processes}.
    \begin{verbatim}
        void lock(int *locked) {
            while (1) {
                while (1) {
                    xchg(locked, &val)
    
                    if(val == 0)
                        break;
                    else
                        sleep();
                }
            }

            void release(int *locked) {
                *locked = 0;
                wakeup();
            }
        }
    \end{verbatim}
    \begin{itemize}
        \item This is more efficient, as it does not waste the CPU's time.
        \item There is a problem with this solution, known as the \textbf{Thundering Herd Problem}: If there are a large number of processes waiting to access the resource, and wakeup is called, all of the processes wake up at the same time. This leads to a large amount of context-switches. If more processes continue to wait, this could lead to \textbf{starvation}.
        \item The \textbf{Thundering Heard Problem} can be solved by placing the processes waiting to access the resource in a \textbf{queue}.
    \end{itemize}
    \item A \textbf{Semaphore} is mechanism for managing shared memory, that is used to produce and consume data (i.e. to solve the problem of the producer, or consumer being faster).
    \item A \textbf{Semaphore} uses an unsigned integer with two functions: wait, and post. The wait function is used to wait until the counter goes above 0, decrements it and pulls 1 data entry. The post function is used when new data is being posted, and increments the counter.
    \begin{verbatim}
        void wait(int *S) {
            while (*S <= 0);
            *S--;
        }

        void post(int *S) {
            *S++;
        }
    \end{verbatim}
    \item \textbf{Semaphores are atomic}, if one \textbf{process} or \textbf{thread} increments the integer, and another decrements, the two operations \textbf{cannot interrupt each other}.
    \item We can use \textbf{semaphores} to solve the \textbf{critical section probelm} by having a \textbf{binary semaphore}, but is it not recommended because anythread can unlock it instad of the thread executing it's critical section.
\end{itemize}

\end{document}
