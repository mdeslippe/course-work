\documentclass{article}
\usepackage[english]{babel}
\usepackage{longtable}
\usepackage[top=1in, bottom=0.25in, left=1.25in, right=1.25in,includefoot,heightrounded]{geometry}
\usepackage{indentfirst}
\usepackage[utf8]{inputenc}
\usepackage{amsmath,amssymb}
\usepackage{graphicx,tikz}
\usepackage{hyperref}
\usepackage[colorinlistoftodos]{todonotes}
\usepackage[document]{ragged2e}
\usepackage{fancyhdr}
\usepackage{enumerate}
\usepackage{listings}
\usepackage{color}
\usepackage{flowchart}
\usepackage{graphicx}
\usepackage{listings}
\usepackage{xcolor}

\definecolor{codegreen}{rgb}{0,0.6,0}
\definecolor{codegray}{rgb}{0.5,0.5,0.5}
\definecolor{codepurple}{rgb}{0.58,0,0.82}
\definecolor{backcolour}{rgb}{0.95,0.95,0.92}

\lstdefinestyle{mystyle}{
    backgroundcolor=\color{backcolour},   
    commentstyle=\color{codegreen},
    keywordstyle=\color{magenta},
    numberstyle=\tiny\color{codegray},
    stringstyle=\color{codepurple},
    basicstyle=\ttfamily\footnotesize,
    breakatwhitespace=false,         
    breaklines=true,                 
    captionpos=b,                    
    keepspaces=true,                 
    numbers=left,                    
    numbersep=5pt,                  
    showspaces=false,                
    showstringspaces=false,
    showtabs=false,                  
    tabsize=2
}

\lstset{style=mystyle}

\usetikzlibrary{arrows}

\usetikzlibrary{shapes.geometric, arrows}
\tikzstyle{startstop} = [rectangle, rounded corners, minimum width=3cm, minimum height=1cm,text centered, draw=black, fill=red!30]
\tikzstyle{decision} = [diamond, minimum width=4cm, minimum height=0.5cm, text centered, draw=black, fill=green!30]
\tikzstyle{process} = [rectangle, minimum width=3cm, minimum height=1cm, text centered, draw=black, fill=orange!30]
\tikzstyle{arrow} = [thick,->,>=stealth]
\tikzstyle{io} = [trapezium, trapezium left angle=70, trapezium right angle=110, minimum width=2cm, text width=4cm, minimum height=1cm, text centered, draw=black, fill=blue!30]

\pagestyle{fancy}
\fancyhf{}
\lhead{Myles Deslippe}
\rhead{React}
\cfoot{\thepage}

\lstset{inputencoding=ansinew}
\lstset{breaklines=true} 

\begin{document}

    \section*{\centering{Introduction to React}}

    \subsection*{React}
    \begin{itemize}
        \item \textbf{React} is a \textbf{JavaScript library} for creating \textbf{user interfaces}. React was created by \textbf{Facebook}.
        \item \textbf{React} supports \textbf{web applications} via React, and \textbf{native applications} via React Native.
        \item \textbf{React documentation} can be found at \url{https://reactjs.org/}.
    \end{itemize}

    \subsection*{Project Structure}
    \begin{itemize}
        \item \textbf{React projects} consist of \textbf{two main directories}:
        \begin{enumerate}
            \item \textbf{public} | The \textbf{public directory} contains \textbf{static context} (html, images, etc) that \textbf{webpack will not process}.
            \item \textbf{src} | The \textbf{src directory} contains the \textbf{JavaScript code} that \textbf{will be processes by webpack}.
        \end{enumerate}
        \item Inside the \textbf{public directory}, there is a file called \textbf{index.html}, this file is the \textbf{entry point} of the \textbf{webpage}. A minimalistic example of this document is:
        \begin{lstlisting}[language=HTML]

    <!DOCTYPE html>
    <html lang="en">

        <head>
            <title>React App</title>
        </head>

        <body>
            <noscript>
                You need to enable JavaScript to run this app.
            </noscript>
            <div id="root">
            </div>
        </body>

    </html>
        \end{lstlisting}
        \begin{itemize}
            \item Inside this HTML document, there will be a \textbf{div (typically with id=``root'')} that you will use to \textbf{inject elements} with \textbf{react}.
        \end{itemize}
        \item Inside the \textbf{src directory} there will be a \textbf{JavaScript file (typically named ``index.js'')} that will serve as the \textbf{entry point} for the code bundled by \textbf{webpack}. A minimalistic example of this document is:
        \begin{lstlisting}[language=Java]

    import React from "react";
    import ReactDom from "react-dom/client";

    const rootElement = document.getElementById("root");
    const root = ReactDom.createRoot(rootElement);

    root.render(
    );
        \end{lstlisting}
        \begin{itemize}
            \item Inside the \textbf{ReactDom root element} is where \textbf{elements will be rendered from}.
        \end{itemize}
        \item Another important file is the \textbf{package.json file} that is in the same directory as \textbf{public and src}. This file is not specific to React, rather NodeJS. This file \textbf{defines metadata about the project}.
    \end{itemize}    

    \subsection*{Adding Elements to the Page}
    \begin{itemize}
        \item One way you can \textbf{add an element to the page} is with the \textbf{createElement} function:
        \begin{lstlisting}[language=Java]
            
    // Arguments are: Element Tag Name, Properties, Inner HTML
    React.createElement("h1", null, "Hello, World!");
        \end{lstlisting}
        \begin{itemize}
            \item This way of \textbf{creating elements} can become very confusing when other elements are \textbf{nested}.
        \end{itemize}
        \item A more popular way to create elements is with the \textbf{JavaScript XML (JSX) syntax}:
            \begin{lstlisting}[language=HTML]

    <h1>Hello, World!</h1>
            \end{lstlisting}
            \begin{itemize}
                \item Behind the scenes \textbf{babel (a JavaScript ``compiler'') will convert} \textbf{JSX} to a \textbf{createElement function call}.
                \item To use \textbf{JavaScript code inside JSX elements}, you have to \textbf{wrap it} in a \textbf{pair of curly braces}.
            \end{itemize}
    \end{itemize}

    \section*{\centering{React Components}}

    \subsection*{Creating Custom Components}
    \begin{itemize}
        \item A \textbf{component} is a \textbf{JavaScript function or class} that returns \textbf{JSX}.
        \begin{itemize}
            \item Only \textbf{one element / component can be returned}, however they \textbf{can contain nested elements / components}.
        \end{itemize}
        \item \textbf{Components} are reusable.
        \item The \textbf{naming convention for components} is \textbf{pascal case}.
        \item An example component is:
        \begin{lstlisting}[language=Java]
            
    // Defining the component.
    function MyComponent() {
        return (
            <h1>This is my component</h1>
        );
    }
        \end{lstlisting}
        \item There are \textbf{two ways} to \textbf{use components}:
        \begin{lstlisting}[language=Java]

    // The first way is with self closing tags.
    <MyComponent />

    // The second way is with opening and closing tags.
    <MyComponent></MyComponent>
        \end{lstlisting}
        \begin{itemize}
            \item \textbf{Opening and closing tags} are typically used \textbf{if the component has nested elements / components}. Other than that, they do the same thing.
        \end{itemize}
    \end{itemize}

    \subsection*{React Fragments}
    \begin{itemize}
        \item It is possible to \textbf{render several elements} from a \textbf{``single component''} using \textbf{fragments}.
        \item \textbf{Fragments} are an \textbf{empty component} that \textbf{only renders its children}.
        \item There are two ways to do this:
        \begin{lstlisting}[language=Java]

    // The first way is with the React.Fragment component
    function MyComponent() {
        return (
            <React.Fragment>
                // Elements and components
            </React.Fragment>
        )
    }

    // The second way is with the empty component
    function MyComponent() {
        return (
            <>
                // Element / component list.
            </>
        )
    }
        \end{lstlisting}
    \end{itemize}

    \subsection*{Component Properties}
    \begin{itemize}
        \item To make \textbf{components more dynamic and reusable} we can pass \textbf{properties to components} to change the \textbf{content rendered}.
        \item Using the \textbf{JSX syntax}, you can use \textbf{key-value pairs} the same way you would with \textbf{regular HTML} to pass properties.
        \item To \textbf{receive the properties} in the \textbf{component's definition}, you add a \textbf{props parameter} which will receive the key-value pairs as an object.
        \item For example:
        \begin{lstlisting}[language=Java]

    // Component definition.
    function MyNumber(props) {
        return (
            <p>My number is {props.number}!</p>
        )
    }

    // Rending the component.
    <MyNumber number={3} />
        \end{lstlisting}
        \item When \textbf{dynamically rendering a list} you \textbf{MUST ALWAYS} give \textbf{each element in the list a ``key'' property that is unique} (the index of each element in the list is not a good key, it should be some type of unique immutable id).
    \end{itemize}

\end{document}