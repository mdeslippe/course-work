\documentclass{article}
\usepackage[english]{babel}
\usepackage{longtable}
\usepackage[top=1in, bottom=0.25in, left=1.25in, right=1.25in,includefoot,heightrounded]{geometry}
\usepackage{indentfirst}
\usepackage[utf8]{inputenc}
\usepackage{amsmath,amssymb}
\usepackage{graphicx,tikz}
\usepackage{hyperref}
\usepackage[colorinlistoftodos]{todonotes}
\usepackage[document]{ragged2e}
\usepackage{fancyhdr}
\usepackage{enumerate}
\usepackage{listings}
\usepackage{color}
\usepackage{flowchart}
\usepackage{graphicx}
\usepackage{listings}
\usepackage{xcolor}

\definecolor{codegreen}{rgb}{0,0.6,0}
\definecolor{codegray}{rgb}{0.5,0.5,0.5}
\definecolor{codepurple}{rgb}{0.58,0,0.82}
\definecolor{backcolour}{rgb}{0.95,0.95,0.92}

\lstdefinestyle{mystyle}{
    backgroundcolor=\color{backcolour},   
    commentstyle=\color{codegreen},
    keywordstyle=\color{magenta},
    numberstyle=\tiny\color{codegray},
    stringstyle=\color{codepurple},
    basicstyle=\ttfamily\footnotesize,
    breakatwhitespace=false,         
    breaklines=true,                 
    captionpos=b,                    
    keepspaces=true,                 
    numbers=left,                    
    numbersep=5pt,                  
    showspaces=false,                
    showstringspaces=false,
    showtabs=false,                  
    tabsize=2
}

\lstset{style=mystyle}

\usetikzlibrary{arrows}

\usetikzlibrary{shapes.geometric, arrows}
\tikzstyle{startstop} = [rectangle, rounded corners, minimum width=3cm, minimum height=1cm,text centered, draw=black, fill=red!30]
\tikzstyle{decision} = [diamond, minimum width=4cm, minimum height=0.5cm, text centered, draw=black, fill=green!30]
\tikzstyle{process} = [rectangle, minimum width=3cm, minimum height=1cm, text centered, draw=black, fill=orange!30]
\tikzstyle{arrow} = [thick,->,>=stealth]
\tikzstyle{io} = [trapezium, trapezium left angle=70, trapezium right angle=110, minimum width=2cm, text width=4cm, minimum height=1cm, text centered, draw=black, fill=blue!30]

\pagestyle{fancy}
\fancyhf{}
\lhead{Myles Deslippe}
\rhead{React}
\cfoot{\thepage}

\lstset{inputencoding=ansinew}
\lstset{breaklines=true} 

\begin{document}

    \section*{\centering{Introduction to React}}

    \subsection*{React}
    \begin{itemize}
        \item \textbf{React} is a \textbf{JavaScript library} for creating \textbf{user interfaces}. React was created by \textbf{Facebook}.
        \item \textbf{React} supports \textbf{web applications} via React, and \textbf{native applications} via React Native.
        \item \textbf{React documentation} can be found at \url{https://reactjs.org/}.
    \end{itemize}

    \subsection*{Project Structure}
    \begin{itemize}
        \item \textbf{React projects} consist of \textbf{two main directories}:
        \begin{enumerate}
            \item \textbf{public} | The \textbf{public directory} contains \textbf{static context} (html, images, etc) that \textbf{webpack will not process}.
            \item \textbf{src} | The \textbf{src directory} contains the \textbf{JavaScript code} that \textbf{will be processes by webpack}.
        \end{enumerate}
        \item Inside the \textbf{public directory}, there is a file called \textbf{index.html}, this file is the \textbf{entry point} of the \textbf{webpage}. A minimalistic example of this document is:
        \begin{lstlisting}[language=HTML]

    <!DOCTYPE html>
    <html lang="en">

        <head>
            <title>React App</title>
        </head>

        <body>
            <noscript>
                You need to enable JavaScript to run this app.
            </noscript>
            <div id="root">
            </div>
        </body>

    </html>
        \end{lstlisting}
        \begin{itemize}
            \item Inside this HTML document, there will be a \textbf{div (typically with id=``root'')} that you will use to \textbf{inject elements} with \textbf{react}.
        \end{itemize}
        \item Inside the \textbf{src directory} there will be a \textbf{JavaScript file (typically named ``index.js'')} that will serve as the \textbf{entry point} for the code bundled by \textbf{webpack}. A minimalistic example of this document is:
        \begin{lstlisting}[language=Java]

    import React from "react";
    import ReactDom from "react-dom/client";

    const rootElement = document.getElementById("root");
    const root = ReactDom.createRoot(rootElement);

    root.render(
    );
        \end{lstlisting}
        \begin{itemize}
            \item Inside the \textbf{ReactDom root element} is where \textbf{elements will be rendered from}.
        \end{itemize}
        \item Another important file is the \textbf{package.json file} that is in the same directory as \textbf{public and src}. This file is not specific to React, rather NodeJS. This file \textbf{defines metadata about the project}.
    \end{itemize}    

    \subsection*{Adding Elements to the Page}
    \begin{itemize}
        \item One way you can \textbf{add an element to the page} is with the \textbf{createElement} function:
        \begin{lstlisting}[language=Java]
            
    // Arguments are: Element Tag Name, Properties, Inner HTML
    React.createElement("h1", null, "Hello, World!");
        \end{lstlisting}
        \begin{itemize}
            \item This way of \textbf{creating elements} can become very confusing when other elements are \textbf{nested}.
        \end{itemize}
        \item A more popular way to create elements is with the \textbf{JavaScript XML (JSX) syntax}:
            \begin{lstlisting}[language=HTML]

    <h1>Hello, World!</h1>
            \end{lstlisting}
            \begin{itemize}
                \item Behind the scenes \textbf{babel (a JavaScript ``compiler'') will convert} \textbf{JSX} to a \textbf{createElement function call}.
                \item To use \textbf{JavaScript code inside JSX elements}, you have to \textbf{wrap it} in a \textbf{pair of curly braces}.
            \end{itemize}
    \end{itemize}

    \section*{\centering{React Components}}

    \subsection*{Creating Custom Components}
    \begin{itemize}
        \item A \textbf{component} is a \textbf{JavaScript function or class} that returns \textbf{JSX}.
        \begin{itemize}
            \item Only \textbf{one element / component can be returned}, however they \textbf{can contain nested elements / components}.
        \end{itemize}
        \item \textbf{Components} are reusable.
        \item The \textbf{naming convention for components} is \textbf{pascal case}.
        \item An example component is:
        \begin{lstlisting}[language=Java]
            
    // Defining the component.
    function MyComponent() {
        return (
            <h1>This is my component</h1>
        );
    }
        \end{lstlisting}
        \item There are \textbf{two ways} to \textbf{use components}:
        \begin{lstlisting}[language=Java]

    // The first way is with self closing tags.
    <MyComponent />

    // The second way is with opening and closing tags.
    <MyComponent></MyComponent>
        \end{lstlisting}
        \begin{itemize}
            \item \textbf{Opening and closing tags} are typically used \textbf{if the component has nested elements / components}. Other than that, they do the same thing.
        \end{itemize}
    \end{itemize}

    \subsection*{React Fragments}
    \begin{itemize}
        \item It is possible to \textbf{render several elements} from a \textbf{``single component''} using \textbf{fragments}.
        \item \textbf{Fragments} are an \textbf{empty component} that \textbf{only renders its children}.
        \item There are two ways to do this:
        \begin{lstlisting}[language=Java]

    // The first way is with the React.Fragment component
    function MyComponent() {
        return (
            <React.Fragment>
                // Elements and components
            </React.Fragment>
        )
    }

    // The second way is with the empty component
    function MyComponent() {
        return (
            <>
                // Element / component list.
            </>
        )
    }
        \end{lstlisting}
    \end{itemize}

    \subsection*{Component Properties}
    \begin{itemize}
        \item To make \textbf{components more dynamic and reusable} we can pass \textbf{properties to components} to change the \textbf{content rendered}.
        \item Using the \textbf{JSX syntax}, you can use \textbf{key-value pairs} the same way you would with \textbf{regular HTML} to pass properties.
        \item To \textbf{receive the properties} in the \textbf{component's definition}, you add a \textbf{props parameter} which will receive the key-value pairs as an object.
        \item For example:
        \begin{lstlisting}[language=Java]

    // Component definition.
    function MyNumber(props) {
        return (
            <p>My number is {props.number}!</p>
        )
    }

    // Rending the component.
    <MyNumber number={3} />
        \end{lstlisting}
        \item When \textbf{dynamically rendering a list} you \textbf{MUST ALWAYS} give \textbf{each element in the list a ``key'' property that is unique} (the index of each element in the list is not a good key, it should be some type of unique immutable id).
    \end{itemize}

    \section*{\centering{Using State and Side Effects in React}}

    \subsection*{Using State in React}
    \begin{itemize}
        \item To \textbf{use state} in a \textbf{functional component}, you can use the \textbf{useState React hook}.
        \item The following example demonstrates how to use state:
        \begin{lstlisting}[language=Java]
    
    import {useState} from 'react';

    function MyComponent () {

        // useState accepts an inital state value as an argument, 
        // and returns an array containing the current state, and a 
        // function to update the state.
        //
        // The function to set the state will receive the previous
        // state as the first parameter, this can be used to update
        // the state if necessary.
        const [state, setState] = useState(initialStateValue);

        // ...
    }
        \end{lstlisting}
    \end{itemize}

    \subsection*{Side Effects in React}
    \begin{itemize}
        \item When performing \textbf{actions that have side effects} (or actions that are not involved in the rendering process) in \textbf{functional components}, the \textbf{useEffect hook should be used}.
        \item The following example demonstrates how to use an effect:
        \begin{lstlisting}[language=Java]
    
    import {useEffect} from 'react';

    function MyComponent() {

        // The code that performs the action with side effects or the
        // action that is not involved in the rendering process is the
        // first arugment provided in useEffect (a function must be passed).
        // The second argument is a dependency array, anytime the dependency
        // array values are modified, the useEffect hook will execute the
        // function again.
        //
        // If the dependency array provided is empty, the effect will only 
        // be called once.
        useEffect(() => {
            // Perform side effects here.
        }, dependencyArray);

        // ...
    }
        \end{lstlisting}
    \end{itemize}

    \subsection*{React Reducer Hook}
    \begin{itemize}
        \item Similar to the \textbf{useState hook} the \textbf{useReducer hook} allows us to create state, but provides the ability to automatically update the state with a \textbf{predetermined function}.
        \item The following example demonstrates how to use a reducer:
        \begin{lstlisting}[language=Java]
            
    import {useReducer} from 'react';

    function MyComponent() {

        // The reducer returns an array containing the current state, and
        // a function to update the state based on the predetermined function.
        // The first argument in useReducer is the function to update the state,
        // the second argument is the inital state.
        const [count, updateCount] = useReducer((count => count + 1), 0);

        // We can then do the following:
        updateCount();

        // ...
    }
        \end{lstlisting}
    \end{itemize}

    \section*{\centering{Handling Forms in React}}

    \subsection*{Uncontrolled and Controlled Components}
    \begin{itemize}
        \item An \textbf{uncontrolled component} is a \textbf{component} that \textbf{renders form elements} such that the \textbf{element's data} is \textbf{managed by the DOM} (the default DOM behavior).
        \item A \textbf{controlled component} is a \textbf{component} that \textbf{renders form elements} such that the \textbf{elements data} is stored in the \textbf{form component's state}.
        \item The following is an example of a \textbf{controlled component}:
        \begin{lstlisting}[language=Java]
            
    import {useState} from 'react';

    function MyComponent() {

        const [currentValue, setValue] = useState("");

        return (
            <input 
                type="text"
                value={currentValue}
                onChange={event => setValue(event.target.value)}
            />
        )

    }
        \end{lstlisting}
    \end{itemize}

    \subsection*{Form Libraries}
    \begin{itemize}
        \item There are many \textbf{existing form libraries} that exist to make \textbf{form development easier}. You can consider using them.
        \item Some libraries can be found at: 
        \begin{enumerate}
            \item \url{https://formik.org/}
            \item \url{https://react-hook-form.com/}
            \item \url{https://usehooks.com/}
        \end{enumerate}
    \end{itemize}

    \section*{\centering{Custom Hooks in React}}

    \subsection*{Custom Hooks in React}
    \begin{itemize}
        \item A \textbf{custom hook} is a \textbf{function} (that starts with use in its identifier by convention).
        \item You can then define the hook inside the function.
        \item Inside the custom hook, you are able to use other hooks.
    \end{itemize}

    \section*{\centering{React Router}}

    \subsection*{Introduction to React Router}
    \begin{itemize}
        \item \textbf{React Router} is a \textbf{standard library for routing in React}. It enables the navigation among views of various components in \textbf{React applications}.
        \item To install the \textbf{React Router} you run the following command:
        \begin{lstlisting}[language=bash]

    npm install react-router-dom
        \end{lstlisting}
    \end{itemize}

    \subsection*{Configuring the React Router}
    \begin{itemize}
        \item The following is an example of how to configure the router:
        \begin{lstlisting}[language=Java]
    
    import React from 'react';
    import ReactDOM from 'react-dom';
    import {BrowserRouter, Routes, Route} from 'react-router-dom';
    import {Page1, Page2, Page3} from './Pages';

    ReactDom.render(
        <BrowserRouter>
            <Routes>
                <Route path="/page1" element={<Page1 />}/>
                <Route path="/page2" element={<Page2 />}/>
                <Route path="/page3" element={<Page3 />}/>
            </Routes>
        </BrowserRouter>,
        document.getElementById('root')
    );
        \end{lstlisting}
        \item Note that the routes do not need to be at the top of the component tree, they can appear anywhere.
    \end{itemize}

    \subsection*{Linking React Router Pages}
    \begin{itemize}
        \item To \textbf{link pages together} you can use the \textbf{Link} component.
        \item The following is an example of how to use the Link component:
        \begin{lstlisting}[language=Java]

            import {Link} from 'react-router-dom';

            // ...

            <Link to="path">Name</Link>
        \end{lstlisting}
    \end{itemize}

    \section*{\centering{React Testing and Deployment}}

    \subsection*{Testing Small Functions with Jest}
    \begin{itemize}
        \item When you \textbf{install React} with \textbf{create-react-app}, a \textbf{test script is created} (this script uses the testing library Jest). This script will \textbf{run the test cases you create}.
        \item To \textbf{create a test file}, you make a file in the application with \textbf{the file extension ``.test.js''}.
        \item The \textbf{naming convention} for testing files is to \textbf{give the file the same name as the one you are testing}, the only difference is the \textbf{file extension}.
        \item The following example demonstrates how to create test cases inside of the test files:
        \begin{lstlisting}[language=Java]
            
    test("Descriptive Case Name", () => {
        // Create test environment.
        
        // You can use expect() to perform assertions, if the assertions 
        // are true, the test passes, if false the test fails.
        // There is no limit to the amount of assertions you can have.
        //
        // The toBe function that is returned by the expect function is
        // one of many jest matchers that can be used for assertions.
        expect(thingThatIsBeingTest(args...)).toBe(result);
    })
        \end{lstlisting}
    \end{itemize}

    \subsection*{The React Testing Library}
    \begin{itemize}
        \item When you \textbf{install React} with \textbf{create-react-app}, the \textbf{React Testing Library} is also installed.
        \item When writing \textbf{tests} that \textbf{involve rendering components}, you can use the \textbf{React testing library} to verify they were \textbf{rendered correctly}.
        \item The following example demonstrates how to test the rendering of a component:
        \begin{lstlisting}[language=Java]
            
    import {render} from '@testing-library/react'
    import MyComponent from './MyComponent';

    test("render h1", () => {
        // Creates a react testing library query.
        const {getByText} = render(<MyComponent />);
        const h1 = getByText(/my text/);

        // You can then perform normal jest assertions.
        expect(h1).toHaveTextContent("my text");
    })
        \end{lstlisting}
    \end{itemize}

    \subsection*{Testing Events}
    \begin{itemize}
        \item You can also \textbf{test events} with the \textbf{React Testing Library}.
        \item The following example demonstrates how to test an even:
        \item[] The component (Checkbox.js):
        \begin{lstlisting}[language=Java]
    
    import {useReducer}
    
    export function Checkbox() {
        const [checked, setChecked] = useReducer((checked => !checked), false);

        return (
            <>
                <label htmlFor="myCheckBox">
                    {checked ? "checked" : "not checked"}
                </label>
                <input
                    id="myCheckBox"
                    type="checkbox"
                    value={checked}
                    onChange={setChecked}
                >
            </>
        )
    }
        \end{lstlisting}
        \item[] The test (Checkbox.test.js):
        \begin{lstlisting}[language=Java]
    
    import {render, fireEvent} from '@testing-library/react';
    import {Checkbox} from './Checkbox';

    test("Checkbox component check change event", () => {
        const {getByLabelText} = render(<Checkbox />);
        const checkbox = getByLabelText(/not checked/i);

        // Fire a click event.
        fireEvent.click(checkbox);

        // Assertion
        expect(checkbox.checked).toEqual(true);
    });
        \end{lstlisting}
    \end{itemize}

    \subsection*{Building React Projects for Production}
    \begin{itemize}
        \item To \textbf{build the project} for \textbf{production}, you can use the \textbf{build script}:
        \begin{lstlisting}[language=Bash]
            
    npm run build
        \end{lstlisting}
        \item This build can then be deployed to a server.
    \end{itemize}

\end{document}