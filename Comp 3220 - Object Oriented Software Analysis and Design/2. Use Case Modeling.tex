\documentclass[16pt]{article}
\usepackage[english]{babel}
\usepackage{longtable}
\usepackage[top=1in, bottom=0.25in, left=1.25in, right=1.25in,includefoot,heightrounded]{geometry}
\usepackage{indentfirst}
\usepackage[utf8]{inputenc}
\usepackage{amsmath,amssymb}
\usepackage{graphicx,tikz}
\usepackage{hyperref}
\usepackage[colorinlistoftodos]{todonotes}
\usepackage[document]{ragged2e}
\usepackage{fancyhdr}
\usepackage{enumerate}
\usepackage{listings}
\usepackage{color}
\usepackage{flowchart}
\usepackage{hyperref}
\usetikzlibrary{arrows}

\usetikzlibrary{shapes.geometric, arrows}
\tikzstyle{startstop} = [rectangle, rounded corners, minimum width=3cm, minimum height=1cm,text centered, draw=black, fill=red!30]
\tikzstyle{decision} = [diamond, minimum width=4cm, minimum height=0.5cm, text centered, draw=black, fill=green!30]
\tikzstyle{process} = [rectangle, minimum width=3cm, minimum height=1cm, text centered, draw=black, fill=orange!30]
\tikzstyle{arrow} = [thick,->,>=stealth]
\tikzstyle{io} = [trapezium, trapezium left angle=70, trapezium right angle=110, minimum width=2cm, text width=4cm, minimum height=1cm, text centered, draw=black, fill=blue!30]

\pagestyle{fancy}
\fancyhf{}
\lhead{Myles Deslippe}
\rhead{Comp 3220 | Object Oriented Software Analysis and Design}
\cfoot{\thepage}

\definecolor{MyDarkGreen}{rgb}{0.0,0.4,0.0}
\lstset{inputencoding=ansinew}
\lstset{breaklines=true} 

\begin{document}

    \section*{\centering{Use Cases}}

    \subsection*{Software Requirements}
    \begin{itemize}
        \item A \textbf{software requirement} is:
        \begin{enumerate}
            \item A \textbf{condition or capability} needed by a \textbf{user} to \textbf{solve a problem} or \textbf{achieve an objective}.
            \item A \textbf{condition or capability} that must be \textbf{satisfied by a system}.
            \item A \textbf{documented representation} of a \textbf{condition or capability} as described in (1) and (2).
        \end{enumerate}
        \item A \textbf{functional requirement} is a requirement that \textbf{describes what a system must do} including \textbf{processes, interfaces, and data}.
        \begin{itemize}
            \item \textbf{Function requirements} are described in \textbf{use case documents}, and are modelled in \textbf{OOA} with \textbf{use case diagrams, class diagrams, and interaction diagrams}.
        \end{itemize}
        \item A \textbf{non-functional requirement} is a requirement that \textbf{specifies how the system must perform} including \textbf{response time, security considerations, and the volume of data}.
        \begin{itemize}
            \item \textbf{Non-functional requirements} are documented in a \textbf{requirement list}.
        \end{itemize}
        \item A \textbf{usability requirement} is a requirement that is concerned with \textbf{matching the system to the way people work}.
            \begin{itemize}
                \item \textbf{Usability requirements} measure objectives, including \textbf{characteristics of users, tasks users undertake, situational factors, and the acceptance of criteria for the working system}.
                \item \textbf{Usability requirements} are documented in the \textbf{list of requirements} and may be tested by \textbf{prototypes}.
            \end{itemize}
    \end{itemize}

    \subsection*{Techniques for Finding Requirements}
    \begin{itemize}
        \item There are \textbf{several techniques for finding requirements}.
        \item \textbf{Background reading} is a technique for finding \textbf{requirements} that aims at \textbf{understanding the organization} and it's \textbf{business objectives}.
        \begin{itemize}
            \item Reading material includes \textbf{reports, charts, policies, job descriptions, and existing system documentation}.
            \item This technique \textbf{works best} in the \textbf{initial stage of fact finding} and when the \textbf{analyst is not familia with the organization}.
        \end{itemize}
        \item \textbf{Interviewing} is a technique for getting an \textbf{in-depth understanding} of the \textbf{organization's objectives, and user roles}.
        \begin{itemize}
            \item Interview subjects include \textbf{managers, staff, and customers}.
            \item This technique \textbf{works best} when \textbf{in-depth} information is required. The effectiveness of this technique depends on the skill of the interviewer.
        \end{itemize}
        \item \textbf{Observation} is a technique for find out what \textbf{really happens}, \textbf{not what people say happens}.
        \begin{itemize}
            \item Items to observe include \textbf{what happens to documents, how people carry out processes, quantitative data, a processes from end-to-end}.
        \end{itemize}
        \item \textbf{Document sampling} is a technique for \textbf{providing statistical data} about \textbf{transaction volumes} and \textbf{activity patterns}.
        \begin{itemize}
            \item Document sampling information includes \textbf{copies of empty and completed documents, screenshots of existing systems, numbers of forms filled in, and the lines on the forms}. 
            \item This technique \textbf{works best} when \textbf{error rates are high, large volumes of data are being processes}.
        \end{itemize}
        \item \textbf{Questionnaires} are a technique for obtaining the \textbf{views of a large amount of people in a way that can be analyzed statistically}.
        \begin{itemize}
            \item Questionnaires include \textbf{postal, web-based, and email questionnaires with open-ended and closed-ended questions}. They also \textbf{gather opinions and facts}.
            \item This technique \textbf{works best} when \textbf{staff organizations are geographically dispersed, the system is going to be used by the general public, and when you need to obtain the views of a large amount of people}.
        \end{itemize}
    \end{itemize}

    \section*{\centering{Use Case Descriptions}}

    \subsection*{Use Cases}
    \begin{itemize}
        \item A \textbf{use case} is primarily an \textbf{action of writing text}.
        \item An \textbf{actor} is a \textbf{person or thing} that \textbf{interacts} with the software.
        \item A \textbf{use case} describes \textbf{what happens in the system} when an \textbf{actor uses the software}.
        \item \textbf{Use case modeling} may include a \textbf{use case diagram}; showing the \textbf{name, actors, and relationships} of \textbf{use cases}.
        \item \textbf{Use case development is a key characteristic} of the \textbf{Unified Process}. It serves to help discover \textbf{functional requirements, design construction, test plans, and maintenance to prepare user manuals}.
    \end{itemize}

    \subsection*{Types of Use Case Descriptions}
    \begin{itemize}
        \item There are \textbf{3 types} of \textbf{use case descriptions}:
        \begin{enumerate}
            \item A \textbf{brief use case description} consists of a \textbf{single paragraph} describing the \textbf{main success scenario}.
            \item A \textbf{casual use case description} consists of \textbf{multiple informal paragraphs} covering both the \textbf{main success scenario}, and \textbf{various alternatives}.
            \item A \textbf{fully dressed use case description} consists of a \textbf{detailed description of all steps involved in the main success, alternative, and exception scenarios}. This is usually accompanied by supporting sections, such as pre-conditions and post-conditions.
        \end{enumerate}
        \item The \textbf{fully dressed use case description} contains the following sections:
        \begin{enumerate}
            \item \textbf{The primary actor} - The user who interacts with the system during this use case.
            \item \textbf{Stakeholders and their interests} - The use case covers the functionality that satisfies all the required stakeholder's interests.
            \item \textbf{Pre-conditions} - Conditions that must be true before the main scenario begins without any checking.
            \item \textbf{Post-conditions} - Conditions that must be true on the successful completion of a use case.
            \item \textbf{The main success scenario (detailed)} - The typical path to a successful outcome (describes what needs to happen not how).
            \item \textbf{Alternative flows (detailed)}. - All other paths that may lead to a success or failure.
            \item \textbf{Exceptions (detailed)}. Exceptions that may occur.
            \item \textbf{Special requirements} - Non-functional requirements for the use case.
            \item \textbf{Open issues} - Anything that hs an effect on the functionality of the use case.
        \end{enumerate}
    \end{itemize}

    \section*{\centering{Use Case Diagrams}}

    \subsection*{Models}
    \begin{itemize}
        \item A \textbf{model} is a \textbf{representation of an entity}.
        \item \textbf{Models} can be used in simulations, evolve as we learn, and are quicker to build than the real thing.
        \item A \textbf{useful model} has the \textbf{right level of detail} and represents only what is \textbf{important} for the task.
    \end{itemize}

    \subsection*{Diagrams}
    \begin{itemize}
        \item A \textbf{diagram} is a \textbf{simplified drawing} showing the \textbf{appearance, structure, or workings} of something.
        \item There are \textbf{rules / standards} for drawing \textbf{diagrams}.
    \end{itemize}

    \subsection*{Unified Modeling Language Diagrams}
    \begin{itemize}
        \item The \textbf{Unified Modeling Language (UML)} is a general-purpose \textbf{developmental modeling language} that is intended to provide a standard way to \textbf{visualize a system}.
        \item To model systems based on the \textbf{UML}, we use \textbf{UML diagrams}.
        \item \textbf{UML diagrams} consist of \textbf{icons, 2D symbols, paths, and strings}.
        \item It is more important to \textbf{fully and correctly communicate ideas} than it is to completely adhere to UML notation standards. 
    \end{itemize}

    \subsection*{UML Use Case Diagrams}
    \begin{itemize}
        \item A \textbf{use case diagram} shows the \textbf{names of actors and use cases} along with \textbf{their relationships}.
        \item There are \textbf{four elements} in use case diagrams: \textbf{actors, use cases, subsystem boundaries, and relationships}.
        \item A \textbf{top-level diagram} includes \textbf{top-level use cases} that \textbf{interact directly} with one or more \textbf{actors}.
        \item A \textbf{sub-level diagram} includes a few of the \textbf{top-level} use cases and \textbf{other related use cases}.
        \item \textbf{Actors} in a use case diagram are \textbf{external entities} who \textbf{use the system}.
        \begin{itemize}
            \item \textbf{Primary actors} achieve their goals by using the system.
            \item \textbf{Supporting actors} provide services to the system.
        \end{itemize}        
        \item To \textbf{identify actors} you need to look at \textbf{who will be using the system, and what will they be doing with it}.
    \end{itemize}

    \subsection*{Name and Size of Use Cases}
    \begin{itemize}
        \item A use case describes \textbf{what happens} in the \textbf{system} when it is \textbf{used by an actor}.
        \item The \textbf{name} of a \textbf{use case} is typically a \textbf{verb and a noun}.
        \item The \textbf{size of use cases} should be \textbf{adequate} (not too big or too small).
        \item \textbf{Use cases} focus on \textbf{what, not how}.
    \end{itemize}

    \subsection*{System Boundary}
    \begin{itemize}
        \item A \textbf{use case model} usually consists of \textbf{multiple diagrams}.
        \item The \textbf{boundary separates top-level use cases from actors}. It does not include use cases for external behaviors.
    \end{itemize}

    \subsection*{Entity Relationships}
    \begin{itemize}
        \item To connect entities we use relationships.
        \item The \textbf{include relationship} indicates that \textbf{an entity always uses one or more instances of another entity}.
        \item The \textbf{extend relationship} indicates that \textbf{one use case flows directly from another}.
        \item To draw the \textbf{include and extend relationships in a UML diagram} you use a \textbf{dotted line with arrows pointing towards the entity being included or extended}. The arrow should say "include" or "extend". 
    \end{itemize}

    \subsection*{Creating Use Case Diagrams}
    \begin{itemize}
        \item To create a \textbf{use case diagram}, you do the following:
        \begin{enumerate}
            \item \textbf{Identify actors and uses cases} by reviewing the \textbf{vision document} and the \textbf{list of requirements}.
            \item \textbf{Add elements to high/low level diagrams} by showing the \textbf{system boundaries} as boxes, \textbf{placing primary actors outside of the boxes} and \textbf{primary use cases inside the boxes}. 
            \item \textbf{Refine the diagram} by \textbf{adding use case and actor relationships}, \textbf{adjusting the placement of elements}, and \textbf{linking the use cases with important scenarios}.
        \end{enumerate}
    \end{itemize}

    \subsection*{Activity Diagrams}
    \begin{itemize}
        \item An \textbf{activity diagram} can be used to \textbf{model tasks, describe use-case functionality, describe the logic of an operation, and model the activities that make up the life cycle in the unified process}.
        \item Before drawing you should ask yourself:
        \begin{enumerate}
            \item What is the purpose of the diagram?
            \item What is the name of the use case?
            \item What level of detail is required?
        \end{enumerate}
        \item To create an \textbf{activity diagram} you do the following:
        \begin{enumerate}
            \item Identify actions and their order of flow.
            \item Work on the main flow of actions by
            \begin{itemize}
                \item Creating a start node.
                \item Placing main actions in the order of flow.
                \item Adding a final node at the flow end.
                \item Linking actions with necessary decisions.
                \item Identifying and creating alternative flows.
                \item Introduction fork/joint nodes.
                \item Continuing with other use cases (optional).
            \end{itemize}
            \item Refine the diagram by adding objects and object flows, as well as control flows with IO pins.
            \item To view the \textbf{diagram notation} visit https://www.geeksforgeeks.org/unified-modeling-language-uml-activity-diagrams/.
        \end{enumerate}
    \end{itemize}

\end{document}